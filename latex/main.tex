\documentclass[a4paper]{article}
\title{Standard Code Library}
\author{AntiLeaf}
\date{}

% \usepackage{zxjatype}
% \setjamainfont{ipaexm.ttf}

\usepackage{graphicx, amssymb, amsmath, textcomp, booktabs}
% \usepackage[libertine,vvarbb]{newtxmath}
\usepackage[scr=rsfso]{mathalfa}
%\usepackage[lining,semibold,type1]{libertine} % a bit lighter than Times--no osf in math
\usepackage[T1]{fontenc} % best for Western European languages
\usepackage{minted}
\usepackage{listings, color, setspace, titlesec, fancyhdr, mdframed, multicol}
\usepackage{fontspec}
\usepackage{ucharclasses}
\usepackage{xunicode, xltxtra}
\usepackage{pdfpages}
\usepackage{tocloft}
\usepackage{nameref}
\usepackage{verbatim}
\usepackage{relsize}
\usepackage[normalem]{ulem}

\usepackage{color,xcolor}

\setlength{\itemindent}{0em}
\setlength\parindent{0em}

\definecolor{light-gray}{gray}{0.9}    % 1.灰度
\definecolor{black}{gray}{0.0}

\XeTeXlinebreaklocale "zh"
\XeTeXlinebreakskip = 0pt plus 1pt

%configure space between the two columns
\setlength{\columnsep}{13pt}

%configure fonts
\setmonofont{Consolas}[Scale=0.8]
\newfontfamily\substitutefont{等线}[Scale=0.9]
\setTransitionsForChinese{\begingroup\substitutefont}{\endgroup}

%configure minted to display codes
\definecolor{Gray}{rgb}{0.9,0.9,0.9}

%configure section style of table of content
\renewcommand\cftsecfont{\Large}

%configure section style
\titleformat{\section}
{\huge}			% The style of the section title
{\thesection.}				% a prefix
{4pt}						% How much space exists between the prefix and the title
{}					% How the section is represented
% \titleformat{\section}{\huge}{}{0pt}{}
% \titlespacing{\section}{0pt}{0pt}{0pt}
% \titlespacing{\subsection}{0pt}{0pt}{0pt}
% \titlespacing{\subsubsection}{0pt}{0pt}{0pt}

\usepackage{fancyhdr}
\usepackage[inner=1.35cm, outer=0.9cm, top=1.7cm, bottom=0.4cm]{geometry}

\pagestyle{fancy}
\setlength{\headsep}{0.1cm}
\setlength{\footskip}{0.7cm}

\chead{Standard Code Library}
\cfoot{\thepage}

\renewcommand{\headrulewidth}{0.5pt}
\renewcommand{\footrulewidth}{0.5pt}

\setminted[cpp]{
	style=materiallight,
	mathescape,
	linenos,
	autogobble,
	baselinestretch=0.9,
	tabsize=4,
	fontsize=\normalsize,
	%bgcolor=Gray,
	frame=single,
	framesep=1mm,
	framerule=0.3pt,
	numbersep=1mm,
	breaklines=true,
	breaksymbolsepleft=2pt,
	%breaksymbolleft=\raisebox{0.8ex}{ \small\reflectbox{\carriagereturn}}, %not moe!
	%breaksymbolright=\small\carriagereturn,
	breakbytoken=false,
	showtabs=true,
	tab={\relscale{1.08} $\color{light-gray}{\vert} \ \ \ $ \relscale{1}},
}

\setminted[python]{
	style=material,
	mathescape,
	linenos,
	autogobble,
	baselinestretch=0.9,
	tabsize=4,
	fontsize=\normalsize,
	%bgcolor=Gray,
	frame=single,
	framesep=0.8mm,
	framerule=0.3pt,
	numbersep=0.8mm,
	breaklines=true,
	breaksymbolsepleft=2pt,
	%breaksymbolleft=\raisebox{0.8ex}{ \small\reflectbox{\carriagereturn}}, %not moe!
	%breaksymbolright=\small\carriagereturn,
	breakbytoken=false,
	showtabs=true,
	tab={\relscale{1.08} $\color{light-gray}{\vert} \ \ \ $ \relscale{1}},
}

\begin{document}

	\begin{titlepage}
		
		\includepdf{cover.pdf}

	\end{titlepage}

	\begin{multicols}{2}
		
		\begin{spacing}{1}
			\tableofcontents
		\end{spacing}
		\newpage

	\end{multicols}

	\begin{multicols}{2}

		\section{图论}

			\subsection{最小生成树}
				\subsubsection{Boruvka算法}
					思想:每次选择连接每个连通块的最小边,把连通块缩起来。

每次连通块个数至少减半,所以迭代 $O(\log n)$ 次即可得到最小生成树。

一种比较简单的实现方法:每次迭代遍历所有边,用并查集维护连通性和每个连通块的最小边权。

应用:最小异或生成树

				
				\subsubsection{动态最小生成树}
					\inputminted{cpp}{../src/graph/动态最小生成树.cpp}
				
				% \subsubsection{最小树形图(朱刘算法)}

				
				\subsubsection{Steiner Tree 斯坦纳树}
					{\bfseries 问题}: 一张图上有$k$个关键点, 求让关键点两两连通的最小生成树

{\bfseries 做法}: 状压DP, $f_{i,S}$表示以$i$号点为树根, $i$与$S$中的点连通的最小边权和

转移有两种:
\begin{enumerate}
	\item 枚举子集: $$\begin{aligned}f_{i, S} = \min_{T\subset S} \left\{f_{i, T} + f_{i, S\setminus T}\right\}\end{aligned}$$
	\item 新加一条边: $$\begin{aligned}f_{i, S} = \min_{(i, j)\in E} \left\{f_{j, S} + w_{i, j}\right\}\end{aligned}$$
\end{enumerate}
第一种直接枚举子集DP就行了, 第二种可以用SPFA或者Dijkstra松弛(显然负边一开始全选就行了, 所以只需要处理非负边).

复杂度$O(n 3^k + 2^k m\log n)$.
			
			
			\subsection{最短路}
				% \subsubsection{Dijkstra}
					
				
				% \subsubsection{差分约束}


				\subsubsection{k短路}
					\inputminted{cpp}{../src/graph/k短路.cpp}

			\subsection{仙人掌}
				\subsubsection{仙人掌DP}
					\inputminted{cpp}{../src/graph/仙人掌DP.cpp}
			
			\subsection{二分图}
				% \subsubsection{匈牙利}


				% \subsubsection{Hopcroft-Karp}


				\subsubsection{KM二分图最大权匹配}
					\inputminted{cpp}{../src/graph/KM二分图最大权匹配.cpp}

				% \subsubsection{二分图原理}
			
			
			\subsection{一般图匹配}
				\subsubsection{高斯消元}
					\inputminted{cpp}{../src/graph/基于线性代数的一般图匹配.cpp}

				\subsubsection{带花树}
					\inputminted{cpp}{../src/graph/带花树.cpp}

				\subsubsection{带权带花树}
					(有一说一这玩意实在太难写了, 抄之前建议先想想算法是不是假的或者有SB做法)
					\inputminted{cpp}{../src/graph/带权带花树.cpp}
			
			% \subsection{支配树}
			

			% \subsection{2-SAT}


			\subsection{最大流}
				\subsubsection{Dinic}
					\inputminted{cpp}{../src/graph/Dinic.cpp}

				\subsubsection{ISAP}
					\inputminted{cpp}{../src/graph/ISAP.cpp}

				\subsubsection{HLPP最高标号预流推进}
					\inputminted{cpp}{../src/graph/HLPP.cpp}
			
			\subsection{费用流}
				\subsubsection{SPFA费用流}
					\inputminted{cpp}{../src/graph/SPFA费用流.cpp}

				% \subsubsection{Dijkstra费用流}


				% \subsubsection{zkw费用流}
			

			% \subsection{网络流原理}
				% \subsubsection{最小割}


				% \subsubsection{费用流}


				% \subsubsection{常见建图方法}


				% \subsubsection{例题}
			

			% \subsection{弦图}
				% \subsubsection{完美消除序列、最大势算法}


				% \subsubsection{性质}
			

			% \subsection{其他}
				% \subsubsection{Stoer-Wagner全局最小割}


				% \subsubsection{最小割树}


				% \subsubsection{最大团搜索}
			

		\section{字符串}

			\subsection{AC自动机}
				\inputminted{cpp}{../src/string/AC自动机.cpp}

			\subsection{后缀数组}
				\subsubsection{SA-IS}
					\inputminted{cpp}{../src/string/sais.cpp}
			
				\subsubsection{SAMSA}
				 	\inputminted{cpp}{../src/string/SAMSA.cpp}

			% \subsection{后缀平衡树}


			\subsection{后缀自动机}
				(广义后缀自动机复杂度就是$O\left(n\left|\Sigma\right|\right)$, 也没法做到更低了)
				\inputminted{cpp}{../src/string/后缀自动机.cpp}

			\subsection{回文树}
				\inputminted{cpp}{../src/string/回文树.cpp}

				\subsubsection{广义回文树}
					(代码是梯子剖分的版本,压力不大的题目换成直接倍增就好了,常数只差不到一倍)
					\inputminted{cpp}{../src/string/广义回文树.cpp}


			% \subsection{序列自动机}


			\subsection{Manacher马拉车}
				\inputminted{cpp}{../src/string/manacher.cpp}

			\subsection{KMP}
				

				\subsubsection{ex-KMP}
					\inputminted{cpp}{../src/string/exKMP.cpp}
			
			% \subsection{All Substring LCS}


			% \subsection{字符串原理}


		\section{数学}

			\subsection{插值}
				\subsubsection{牛顿插值}
					牛顿插值的原理是\textbf{二项式反演}.

二项式反演: 

$$ \begin{aligned}
f(n)=\sum_{k=0}^n{n\choose k}g(k)\;\Leftrightarrow\;g(n)=\sum_{k=0}^n\left(-1\right)^{n-k}{n\choose k}f(k)
\end{aligned} $$

可以用$e^x$和$e^{-x}$的麦克劳林展开式证明.

套用二项式反演的结论即可得到牛顿插值: 

$$ \begin{aligned} f(n)=\sum_{i=0}^{k}{n\choose i}r_i \end{aligned} $$
$$ \begin{aligned} r_i=\sum_{j=0}^i(-1)^{i-j}{i\choose j}f(j) \end{aligned} $$

其中$k$表示$f(n)$的最高次项系数.

实现时可以用$k$次差分替代右边的式子:

\begin{minted}{cpp}
for (int i = 0; i <= k; i++)
    r[i] = f(i);
for (int j = 0; j < k; j++)
    for (int i = k; i > j; i--)
        r[i] -= r[i - 1];
\end{minted}

注意到预处理$r_i$的式子满足卷积形式,必要时可以用FFT优化至$O(k\log k)$预处理.

				\subsubsection{拉格朗日插值}
					$$ f(x) = \sum_i f(x_i) \prod_{j \neq i} \frac {x - x_j} {x_i - x_j} $$

快速插值参见 \detailedref{PolyFastInterpolation}。

			
			\subsection{多项式}
				\subsubsection{FFT}
					\inputminted{cpp}{../src/math/FFT.cpp}

				\subsubsection{NTT}
					\inputminted{cpp}{../src/math/NTT.cpp}

				\subsubsection{任意模数卷积(三模数NTT)}
					\inputminted{cpp}{../src/math/任意模数卷积.cpp}

				% \subsection{毛梯梯}

			
				\subsubsection{多项式操作}
					\inputminted{cpp}{../src/math/多项式操作.cpp}

				% \subsubsection{多项式多点求值}


				% \subsubsection{多项式快速插值}


				% \subsubsection{多项式复合与复合逆}

				\subsubsection{拉格朗日反演}
					% 用于求复合逆. \\
如果$f(x)$与$g(x)$互为复合逆, 则有 \\
\begin{math}
\begin{aligned}\relax \left[x^n\right]g(x)=\frac{1}{n}\left[x^{n-1}\right]\left(\frac{x}{f(x)}\right)^n\end{aligned} \\
\begin{aligned}\relax \left[x^n\right]h(g(x))=\frac{1}{n}\left[x^{n-1}\right]h'(x)\left(\frac{x}{f(x)}\right)^n\end{aligned}
\end{math}

				% \subsubsection{分治FFT}


				\subsubsection{半在线卷积}
					\inputminted{cpp}{../src/math/半在线卷积.cpp}

			\subsection{FWT快速沃尔什变换}
				\inputminted{cpp}{../src/math/FWT.cpp}

			\subsection{单纯形}
				\inputminted{cpp}{../src/math/单纯形.cpp}

				% \subsubsection{线性规划对偶原理}


			\subsection{线性代数}
				% \subsubsection{高斯消元}


				% \subsubsection{任意模数高斯消元}


				% \subsubsection{自由元搜索}


				\subsubsection{线性基}
					% \inputminted{cpp}{../src/math/线性基.cpp}


				% \subsubsection{线性代数知识}


			% \subsection{自适应Simpson积分}


			% \subsection{常见结论}
				% \subsubsection{线性齐次递推求通项}
			
				
			\subsection{常见数列}
				\subsubsection{伯努利数}
					\begin{math}
    \begin{aligned}B(x)=\sum_{i\ge 0}\frac{B_i x^i}{i!}=\frac x{e^x-1}\end{aligned} \\
    \begin{aligned}B_n=[n=0]-\sum_{i=0}^{n-1}{n\choose i}\frac{B_i}{n-k+1}\end{aligned} \\
    \begin{aligned}\sum_{i=0}^n{n+1\choose i}B_i=0\end{aligned} \\
    \begin{aligned}S_n(m)=\sum_{i=0}^{m-1}i^n=\sum_{i=0}^n{n\choose i}B_{n-i}\frac{m^{i+1}}{i+1}\end{aligned}
\end{math}


		\section{数论}

			\subsection{$O(n)$预处理逆元}
				\inputminted{cpp}{../src/numbertheory/O(n)求逆元.cpp}

			\subsection{杜教筛}
				\inputminted{cpp}{../src/numbertheory/杜教筛.cpp}
			
			\subsection{线性筛}
				\inputminted{cpp}{../src/numbertheory/扩展线性筛.cpp}

			% \subsection{min_25筛}


			\subsection{Miller-Rabin}
				\inputminted{cpp}{../src/numbertheory/Miller-Rabin.cpp}

			\subsection{Pollard's Rho}
				\inputminted{cpp}{../src/numbertheory/Pollard's Rho.cpp}
			
			% \subsection{组合数取模}


			%\subsection{扩展欧几里得}
				

		\section{数据结构}

			\subsection{线段树}
				\subsubsection{非递归线段树}
					\sout{让 \textit{fstqwq} 手撕}\ 我队友呢?

\begin{itemize}
	\item 如果 $M = 2^k$,则只能维护 $[1, M - 2]$ 范围
	\item 找叶子:$i$ 对应的叶子就是 $i + M$
	\item 单点修改:找到叶子然后向上跳
	\item 区间查询:左右区间各扩展一位,转换成开区间查询
	\inputminted{cpp}{../src/datastructure/非递归线段树单点修改.cpp}

\end{itemize}
	
区间修改要标记永久化,并且求区间和和求最值的代码不太一样。

\textbf{区间加,区间求和}
\inputminted{cpp}{../src/datastructure/非递归线段树区间加区间求和.cpp}

\textbf{区间加,区间求最大值}
\inputminted{cpp}{../src/datastructure/非递归线段树区间加区间求最大值.cpp}


				\subsubsection{主席树}
					参见GREALD07加强版

			\subsection{陈丹琦分治}
				\inputminted{cpp}{../src/datastructure/CDQ分治.cpp}

			% \subsection{整体二分}


			% \subsection{块状链表}


			\subsection{Treap}
				\inputminted{cpp}{../src/datastructure/Treap.cpp}
			
				% \subsubsection{无旋Treap}


				% \subsubsection{可持久化Treap}


			\subsection{Splay}
				(参见LCT,除了\mintinline{cpp}{splay()}需要传一个点表示最终它的父亲,其他写法都和LCT相同)
			
			\subsection{树分治}
				% \subsubsection{静态树分治}

				
				\subsubsection{动态树分治}
					\inputminted{cpp}{../src/datastructure/动态树分治.cpp}

				\subsubsection{紫荆花之恋}
					\inputminted{cpp}{../src/datastructure/紫荆花之恋.cpp}

				\subsection{LCT}
					\subsubsection{不换根(弹飞绵羊)}
						\inputminted{cpp}{../src/datastructure/LCT(不换根).cpp}
				
					\subsubsection{换根/维护生成树(GREALD07加强版)}
						\inputminted{cpp}{../src/datastructure/GREALD07.cpp}

					\subsubsection{维护子树信息}
						\inputminted{cpp}{../src/datastructure/LCT维护子树信息.cpp}
					
					\subsubsection{模板题:动态QTREE4(询问树上相距最远点)}
						\inputminted{cpp}{../src/datastructure/动态QTREE4.cpp}

				% \subsection{K-D树}


				% \subsection{树分块}


				% \subsection{树上莫队}

				\subsection{虚树}
					\inputminted{cpp}{../src/datastructure/虚树.cpp}


				\subsection{长链剖分}
					\inputminted{cpp}{../src/datastructure/长链剖分.cpp}

				\subsection{梯子剖分}
					\inputminted{cpp}{../src/datastructure/梯子剖分.cpp}
				
				\subsection{左偏树}
					(参见k短路)

				\subsection{常见根号思路}
					
\noindent{
	{\large\bfseries{通用}}
	\begin{itemize}
		\item 出现次数大于$\sqrt n$的数不会超过$\sqrt n$个
		\item 对于带修改问题, 如果不方便分治或者二进制分组, 可以考虑对操作分块, 每次查询时暴力最后的$\sqrt n$个修改并更正答案
		\item {\bfseries 根号分治}: 如果分治时每个子问题需要$O(N)$(N是全局问题的大小)的时间, 而规模较小的子问题可以$O(n^2)$解决, 则可以使用根号分治
			\begin{itemize}
				\item 规模大于$\sqrt n$的子问题用$O(N)$的方法解决, 规模小于$\sqrt n$的子问题用$O(n^2)$暴力
				\item 规模大于$\sqrt n$的子问题最多只有$\sqrt n$个
				\item 规模不大于$\sqrt n$的子问题大小的平方和也必定不会超过$n\sqrt n$
			\end{itemize}
		\item 如果输入规模之和不大于$n$(例如给定多个小字符串与大字符串进行询问), 那么规模超过$\sqrt n$的问题最多只有$\sqrt n$个
	\end{itemize}
}

\noindent{
	{\large\bfseries{序列}}
	\begin{itemize}
		\item 某些维护序列的问题可以用分块/块状链表维护
		\item 对于静态区间询问问题, 如果可以快速将左/右端点移动一位, 可以考虑莫队
			\begin{itemize}
				\item 如果强制在线可以分块预处理, 但是一般空间需要$n\sqrt n$ 
					\begin{itemize}
						\item 例题: 询问区间中有几种数出现次数恰好为$k$, 强制在线
					\end{itemize}
				\item 如果带修改可以试着想一想带修莫队, 但是复杂度高达$n^{\frac 5 3}$
			\end{itemize}
		\item 线段树可以解决的问题也可以用分块来做到$O(1)$询问或是$O(1)$修改, 具体要看哪种操作更多
	\end{itemize}
}

\noindent{
	{\large\bfseries{树}}
	\begin{itemize}
		\item 与序列类似, 树上也有树分块和树上莫队
			\begin{itemize}
				\item 树上带修莫队很麻烦, 常数也大, 最好不要先考虑
				\item 树分块不要想当然
			\end{itemize}
		\item 树分治也可以套根号分治, 道理是一样的
	\end{itemize}
}

\noindent{
	{\large\bfseries{字符串}}
	\begin{itemize}
		\item 循环节长度大于$\sqrt n$的子串最多只有$O(n)$个, 如果是极长子串则只有$O(\sqrt n)$个
	\end{itemize}
}

\noindent {
	{\large\bfseries{关于莫队}}

	莫队是可以改造成只有插入和撤销(或者只有删除和撤销)的版本的.

	例如维护dfs序时就可以使用链表, 配合只有删除的莫队就可以做到$O(n\sqrt n)$.

	另外如果$n$和$q$不平衡, 块大小应该设为$\frac n {\sqrt q}$.
}

			\section{动态规划}
				\subsection{决策单调性$O(n\log n)$}
					\inputminted{cpp}{../src/DP/决策单调性.cpp}

			\section{Miscellaneous}
				\subsection{$O(1)$快速乘}
					\inputminted{cpp}{../src/misc/O(1)快速乘.cpp}
				
				\subsection{$O(n^2)$高精度}
					\inputminted{cpp}{../src/misc/高精度.cpp}

				\subsection{xorshift}
					\inputminted{cpp}{../src/misc/xorshift.cpp}
				
				\subsection{枚举子集}
					(注意这是$t\ne 0$的写法, 如果可以等于$0$需要在循环里手动\mintinline{cpp}{break})
\begin{minted}{cpp}
for (int t = s; t; (--t) &= s) {
    // do something
}
\end{minted}
					
				\subsection{STL}
					\noindent

\begin{enumerate}
	\item vector
		\begin{itemize}
			\item \mintinline{cpp}{vector(int nSize)}:创建一个vector,元素个数为nSize
			\item \mintinline{cpp}{vector(int nSize,const t& t)}:创建一个vector,元素个数为nSize,且值均为t
			\item \mintinline{cpp}{vector(begin,end)}:复制[begin,end)区间内另一个数组的元素到vector中
			\item \mintinline{cpp}{void assign(int n,const T& x)}:设置向量中前n个元素的值为x
			\item \mintinline{cpp}{void assign(const_iterator first,const_iterator last)}:向量中[first,last)中元素设置成当前向量元素
		\end{itemize}
	
	\item list
		\begin{itemize}
			\item \mintinline{cpp}{assign()} 给list赋值 
			\item \mintinline{cpp}{back()} 返回最后一个元素 
			\item \mintinline{cpp}{begin()} 返回指向第一个元素的迭代器 
			\item \mintinline{cpp}{clear()} 删除所有元素 
			\item \mintinline{cpp}{empty()} 如果list是空的则返回true 
			\item \mintinline{cpp}{end()} 返回末尾的迭代器
			\item \mintinline{cpp}{erase()} 删除一个元素
			\item \mintinline{cpp}{front()} 返回第一个元素
			\item \mintinline{cpp}{insert()} 插入一个元素到list中
			\item \mintinline{cpp}{max_size()} 返回list能容纳的最大元素数量
			\item \mintinline{cpp}{merge()} 合并两个list
			\item \mintinline{cpp}{pop_back()} 删除最后一个元素
			\item \mintinline{cpp}{pop_front()} 删除第一个元素
			\item \mintinline{cpp}{push_back()} 在list的末尾添加一个元素
			\item \mintinline{cpp}{push_front()} 在list的头部添加一个元素
			\item \mintinline{cpp}{rbegin()} 返回指向第一个元素的逆向迭代器
			\item \mintinline{cpp}{remove()} 从list删除元素
			\item \mintinline{cpp}{remove_if()} 按指定条件删除元素
			\item \mintinline{cpp}{rend()} 指向list末尾的逆向迭代器
			\item \mintinline{cpp}{resize()} 改变list的大小
			\item \mintinline{cpp}{reverse()} 把list的元素倒转
			\item \mintinline{cpp}{size()} 返回list中的元素个数
			\item \mintinline{cpp}{sort()} 给list排序
			\item \mintinline{cpp}{splice()} 合并两个list
			\item \mintinline{cpp}{swap()} 交换两个list
			\item \mintinline{cpp}{unique()} 删除list中重复的元
		\end{itemize}
\end{enumerate}


				\subsection{pb\_ds}


				\subsection{rope}
					

			\section{注意事项}
				\subsection{常见下毒手法}
					\noindent
\begin{itemize}
    \item 0/1base是不是搞混了
    \item 高精度高低位搞反了吗
    \item 线性筛抄对了吗
    \item 快速乘抄对了吗
    \item \mintinline{cpp}{i <= n, j <= m}
    \item sort比较函数是不是比了个寂寞
    \item 该取模的地方都取模了吗
    \item 边界情况(+1-1之类的)有没有想清楚
    \item \bfseries{特判是否有必要, 确定写对了吗}
\end{itemize}

				\subsection{场外相关}
					\input{../src/attention/场外相关.tex}

				\subsection{做题策略与心态调节}
					\noindent
\begin{itemize}
	\item 拿到题后立刻按照商量好的顺序读题, 前半小时最好跳过题意太复杂的题(除非被过穿了)
	
	\item 签到题写完不要激动, 稍微检查一下最可能的下毒点再交, 避免无谓的罚时
		\subitem 一两行的那种傻逼题就算了
		
	\item 读完题及时输出题意, 一方面避免重复读题, 一方面也可以让队友有一个初步印象, 方便之后决定开题顺序
	
	\item 如果不能确定题意就不要贸然输出甚至上机,  尤其是签到题,  因为样例一般都很弱
	
	\item 一个题如果卡了很久又有其他题可以写, 那不妨先放掉写更容易的题, 不要在一棵树上吊死
		\subitem 不要被一两道题搞得心态爆炸, 一方面急也没有意义, 一方面你很可能真的离AC就差一步
		
	\item 榜是不会骗人的, 一个题如果被不少人过了就说明这个题很可能并没有那么难;如果不是有十足的把握就不要轻易开没什么人交的题;另外不要忘记最后一小时会封榜
	
	\item 想不出题/找不出毒自然容易犯困, 一定不要放任自己昏昏欲睡, 最好去洗手间冷静一下, 没有条件就站起来踱步
	
	\item 思考的时候不要挂机, 一定要在草稿纸上画一画, 最好说出声来最不容易断掉思路
	
	\item 出完算法一定要check一下样例和一些trivial的情况, 不然容易写了半天发现写了个假算法
	
	\item 上机前有时间就提前给需要思考怎么写的地方打草稿, 不要浪费机时
	
	\item 查毒时如果最难的地方反复check也没有问题, 就从头到脚仔仔细细查一遍, 不要放过任何细节, 即使是并查集和sort这种东西也不能想当然
	
	\item 后半场如果时间不充裕就不要冒险开难题, 除非真的无事可做
		\subitem 如果是没写过的东西也不要轻举妄动, 在有其他好写的题的时候就等一会再说
		
	\item 大多数时候都要听队长安排, 虽然不一定最正确但可以保持组织性
	
	% \item 最好注意一下影响, 就算忍不住嘴臭也不要太大声
	
	\item 任何时候都不要着急, 着急不能解决问题, 不要当喆国王
	
	\item 输了游戏, 还有人生;赢了游戏, 还有人生.
\end{itemize}

	\end{multicols}

\end{document}
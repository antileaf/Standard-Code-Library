注意Powerful Number筛只能求积性函数的前缀和.

本质上就是构造一个方便求前缀和的函数, 然后做类似杜教筛的操作.

定义Powerful Number表示每个质因子幂次都大于$1$的数, 显然最多有$\sqrt n$个.

设我们要求和的函数是$f(n)$, 构造一个方便求前缀和的\textbf{积性}函数$g(n)$使得$g(p) = f(p)$.

那么就存在一个积性函数$h = f * g ^ {-1}$, 也就是$f = g *h$. 可以证明$h(p) = 0$, 所以只有Powerful Number的$h$值不为0.

$$ \begin{aligned}
	S_f(i) = \sum_{d = 1} ^ n h(d) S_g \left( \left\lfloor \frac n d \right\rfloor \right)
\end{aligned} $$

只需要枚举每个Powerful Number作为$d$, 然后用杜教筛计算$g$的前缀和.

求$h(d)$时要先预处理$h(p^k)$, 显然有

$$ \begin{aligned}
	h \left(p ^ k \right) = f \left(p ^ k \right) - \sum_{i = 1} ^ k g \left( p ^ i \right) h \left( p ^ {k - i} \right)
\end{aligned} $$

处理完之后DFS就行了. (显然只需要筛$\sqrt n$以内的质数.)

复杂度取决于杜教筛的复杂度, 特殊题目构造的好也可以做到$O \left( \sqrt n \right)$.

例题:

\begin{itemize}
	\item $f \left( p ^ k \right) = p ^ k \left( p ^ k - 1 \right)$
		\subitem $g(n) = \text{id}(n) \varphi(n)$.
	\item $f \left( p ^ k \right) = p \, \text{xor} \, k$
		\subitem $n$为偶数时$g(n) = 3 \varphi(n)$, 否则$g(n) = \varphi(n)$.
\end{itemize}
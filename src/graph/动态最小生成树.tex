% 动态最小生成树的离线算法比较容易,而在线算法通常极为复杂。

对时间分治,在每层分治时找出一定在/不在 MST 上的边,只带着不确定边继续递归。

过程中的两种重要操作如下:

\begin{itemize}
	\item Reduction:待修改边标为 $+\infty$,跑 MST 后把非树边删掉,减少无用边
	\item Contraction:待修改边标为 $-\infty$,跑 MST 后把待修改边之外的所有树边缩点,缩掉必须边
\end{itemize}

每轮分治需要 Reduction-Contraction,借此减少不确定边,从而保证复杂度。

复杂度证明:假设当前区间有 $k$ 条待修改边,$n$ 和 $m$ 表示点数和边数,那么最坏情况下 R-C 的效果为 $(n, m) \to (n, n + k - 1) \to (k + 1, 2k)$。

\inputminted{cpp}{../src/graph/动态最小生成树.cpp}

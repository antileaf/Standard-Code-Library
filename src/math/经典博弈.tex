\begin{enumerate}

\item \textbf{阶梯博弈}

台阶的每层都有一些石子,每次可以选一层(但不能是第 $0$ 层),把任意个石子移到低一层。

\paragraph{结论} 奇数层的石子数量进行异或和即可。

实际上只要路径长度唯一就可以,比如在树上博弈,然后石子向根节点方向移动,那么就是奇数深度的石子数量进行异或和。

\item \textbf{可以同时操作多个子游戏}

如果某个游戏由若干个独立的子游戏组成,并且每次可以 \textbf{任意选几个} (当然至少一个)子游戏进行操作,那么结论是:所有子游戏都必败时先手才会必败,否则先手必胜。

\item \textbf{每次最多操作 $k$ 个子游戏 (Nim-K)}

如果每次最多操作 $k$ 个子游戏,结论是:把所有子游戏的 SG 函数写成二进制表示,如果每一位上的 $1$ 个数都是 $(k+1)$ 的倍数,则先手必败,否则先手必胜。

(实际上前一条可以看做 $k = \infty$ 的情况,也就是所有 SG 值都是 $0$ 时才会先手必败。)

如果要求整个游戏的 SG 函数,就按照上面的方法每个二进制位相加后 $\bmod (k+1)$,视为 $(k+1)$ 进制数求值即可。(\textbf{未验证})

\item \textbf{反 Nim 游戏 (Anti-Nim)}

和 Nim 游戏差不多,唯一的不同是取走最后一个石子的输。

分两种情况:

\begin{itemize}
	\item 所有堆石子个数都是 $1$:有偶数堆时先手必胜,否则先手必败。
	\item 存在某个堆石子数多于 $1$:异或和不为 $0$ 则先手必胜,否则先手必败。
\end{itemize}

当然石子个数实际上就是 SG 函数,所以判别条件全都改成 SG 函数也是一样的。(\textbf{未验证})

\item \textbf{威佐夫博弈}

有两堆石子,每次要么从一堆中取任意个,要么从两堆中都取走相同数量。也等价于两个人移动一个只能向左上方走的皇后,不能动的输。

\paragraph{结论}设两堆石子分别有 $a$ 个和 $b$ 个,且 $a < b$,则先手必败当且仅当 $a = \left\lfloor (b - a) \frac {1 + \sqrt 5} 2 \right\rfloor$。

\item \textbf{删子树博弈}

有一棵有根树,两个人轮流操作,每次可以选一个点(除了根节点)然后把它的子树都删掉,不能操作的输。

\paragraph{结论}

$$ SG(u) = \text{XOR} _{v \in son_u} \left( SG(v) + 1 \right) $$

\item \textbf{无向图游戏}

在一个无向图上的某个点上摆一个棋子,两个人轮流把棋子移动到相邻的点,并且每个点只能走一次,不能操作的输。

\paragraph{结论} 如果某个点一定在最大匹配中,则先手必胜,否则先手必败。

\end{enumerate}

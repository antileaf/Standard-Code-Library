KMP和AC自动机的fail指针存储的都是它在串或者字典树上的最长后缀, 因此要判断两个前缀是否互为后缀时可以直接用fail指针判断. 当然它不能做子串问题, 也不能做最长公共后缀.

后缀数组利用的主要是LCP长度可以按照字典序做RMQ的性质, 与某个串的LCP长度$\ge$某个值的后缀形成一个区间. 另外一个比较好用的性质是本质不同的子串个数 = 所有子串数 - 字典序相邻的串的height.

后缀自动机实际上可以接受的是所有后缀, 如果把中间状态也算上的话就是所有子串. 它的fail指针代表的也是当前串的后缀, 不过注意每个状态可以代表很多状态, 只要右端点在right集合中且长度处在$(val_{par_p}, val_p]$中的串都被它代表.

后缀自动机的fail树也就是\textbf{反串}的后缀树. 每个结点代表的串和后缀自动机同理, 两个串的LCP长度也就是他们在后缀树上的LCA.
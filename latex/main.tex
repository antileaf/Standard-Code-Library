\documentclass[a4paper]{article}
\title{Standard Code Library}
\author{AntiLeaf}
\date{}

% green= '#10BC72' # '#C3E88D'
% cyan= '#00A1D6' # '#89DDFF'

% \usepackage{zxjatype}
% \setjamainfont{ipaexm.ttf}

\usepackage{graphicx, amssymb, amsmath, textcomp, booktabs}
% \usepackage[libertine, vvarbb]{newtxmath}
\usepackage[scr=rsfso]{mathalfa}
%\usepackage[lining, semibold, type1]{libertine} % a bit lighter than Times--no osf in math
\usepackage[T1]{fontenc} % best for Western European languages
\usepackage{minted}
\usepackage{listings, color, setspace, titlesec, fancyhdr, mdframed, multicol}
\usepackage{fontspec}
\usepackage{ucharclasses}
\usepackage{xunicode, xltxtra}
\usepackage{pdfpages}
\usepackage{tocloft}
\usepackage{nameref}
\usepackage{verbatim}
\usepackage{relsize}
\usepackage[normalem]{ulem}

\usepackage{color,xcolor}

\setlength{\itemindent}{0em}
\setlength\parindent{0em}

\definecolor{light-gray}{gray}{0.9}    % 1.灰度
\definecolor{black}{gray}{0.0}

\XeTeXlinebreaklocale "zh"
\XeTeXlinebreakskip = 0pt plus 1pt

%configure space between the two columns
\setlength{\columnsep}{13pt}

%configure fonts
\setmonofont{Consolas}[Scale=0.8]
\newfontfamily\substitutefont{等线}[Scale=0.9]
\setTransitionsForChinese{\begingroup\substitutefont}{\endgroup}

%configure minted to display codes
\definecolor{Gray}{rgb}{0.9,0.9,0.9}

%configure section style of table of content
\renewcommand\cftsecfont{\Large}

%configure section style
\titleformat{\section}
{\huge}			% The style of the section title
{\thesection.}				% a prefix
{4pt}						% How much space exists between the prefix and the title
{}					% How the section is represented
% \titleformat{\section}{\huge}{}{0pt}{}
% \titlespacing{\section}{0pt}{0pt}{0pt}
% \titlespacing{\subsection}{0pt}{0pt}{0pt}
% \titlespacing{\subsubsection}{0pt}{0pt}{0pt}

\usepackage{fancyhdr}
\usepackage[inner=1.35cm, outer=0.9cm, top=1.7cm, bottom=0.4cm]{geometry}

\pagestyle{fancy}
\setlength{\headsep}{0.1cm}
\setlength{\footskip}{0.7cm}

\chead{Standard Code Library}
\cfoot{\thepage}

\renewcommand{\headrulewidth}{0.5pt}
\renewcommand{\footrulewidth}{0.5pt}

\setminted[cpp]{
	style=materiallight,
	mathescape,
	linenos,
	autogobble,
	baselinestretch=0.9,
	tabsize=4,
	fontsize=\normalsize,
	%bgcolor=Gray,
	frame=single,
	framesep=1mm,
	framerule=0.3pt,
	numbersep=1mm,
	breaklines=true,
	breaksymbolsepleft=2pt,
	%breaksymbolleft=\raisebox{0.8ex}{ \small\reflectbox{\carriagereturn}}, %not moe!
	%breaksymbolright=\small\carriagereturn,
	breakbytoken=false,
	showtabs=true,
	tab={\relscale{1.08} $\color{light-gray}{\vert} \ \ \ $ \relscale{1}},
}

\setminted[python]{
	style=materiallight,
	mathescape,
	linenos,
	autogobble,
	baselinestretch=0.9,
	tabsize=4,
	fontsize=\normalsize,
	%bgcolor=Gray,
	frame=single,
	framesep=0.8mm,
	framerule=0.3pt,
	numbersep=0.8mm,
	breaklines=true,
	breaksymbolsepleft=2pt,
	%breaksymbolleft=\raisebox{0.8ex}{ \small\reflectbox{\carriagereturn}}, %not moe!
	%breaksymbolright=\small\carriagereturn,
	breakbytoken=false,
	showtabs=true,
	tab={\relscale{1.08} $\color{light-gray}{\vert} \ \ \ $ \relscale{1}},
}

\begin{document}

	\begin{titlepage}
		
		\includepdf{cover.pdf}

	\end{titlepage}

	\begin{multicols}{2}
		
		\begin{spacing}{1}
			\tableofcontents
		\end{spacing}
		\newpage

	\end{multicols}

	\begin{multicols}{2}

		\section{数学}

			\subsection{插值}
				\subsubsection{牛顿插值}
					牛顿插值的原理是\textbf{二项式反演}.

二项式反演: 

$$ \begin{aligned}
f(n)=\sum_{k=0}^n{n\choose k}g(k)\;\Leftrightarrow\;g(n)=\sum_{k=0}^n\left(-1\right)^{n-k}{n\choose k}f(k)
\end{aligned} $$

可以用$e^x$和$e^{-x}$的麦克劳林展开式证明.

套用二项式反演的结论即可得到牛顿插值: 

$$ \begin{aligned} f(n)=\sum_{i=0}^{k}{n\choose i}r_i \end{aligned} $$
$$ \begin{aligned} r_i=\sum_{j=0}^i(-1)^{i-j}{i\choose j}f(j) \end{aligned} $$

其中$k$表示$f(n)$的最高次项系数.

实现时可以用$k$次差分替代右边的式子:

\begin{minted}{cpp}
for (int i = 0; i <= k; i++)
    r[i] = f(i);
for (int j = 0; j < k; j++)
    for (int i = k; i > j; i--)
        r[i] -= r[i - 1];
\end{minted}

注意到预处理$r_i$的式子满足卷积形式,必要时可以用FFT优化至$O(k\log k)$预处理.

				\subsubsection{拉格朗日插值}
					$$ f(x) = \sum_i f(x_i) \prod_{j \neq i} \frac {x - x_j} {x_i - x_j} $$

快速插值参见 \detailedref{PolyFastInterpolation}。

			
			\subsection{多项式}
				\subsubsection{FFT}
					\inputminted{cpp}{../src/math/FFT.cpp}

				\subsubsection{NTT}
					\inputminted{cpp}{../src/math/NTT.cpp}

				\subsubsection{任意模数卷积}
					三模数 NTT 和直接拆系数 FFT 都太慢了,不要用。

MTT 的原理就是拆系数 FFT,只不过优化了做变换的次数。

考虑要对 $A(x), B(x)$ 两个多项式做 DFT,可以构造两个复多项式

$$ P(x) = A(x) + iB(x) \quad Q(x) = A(x) - iB(x) $$

只需要 DFT 一个,另一个 DFT 实际上就是前者反转再取共轭,再利用

$$ A(x) = \frac {P(x) + Q(x)} 2 \quad B(x) = \frac {P(x) - Q(x)} {2i} $$

即可还原出 $A(x), B(x)$。

IDFT 的道理更简单,如果要对 $A(x)$ 和 $B(x)$ 做 IDFT,只需要对 $A(x) + i B(x)$ 做 IDFT 即可,因为 IDFT 的结果必定为实数,所以结果的实部和虚部就分别是 $A(x)$ 和 $B(x)$。

\textbf{实际上任何同时对两个实序列进行 DFT,或者同时对结果为实序列的 DFT 进行逆变换时都可以按照上面的方法优化,可以减少一半的 DFT 次数。}

\inputminted{cpp}{../src/math/MTT.cpp}


				% \subsection{毛梯梯}

			
				\subsubsection{多项式操作}
					\inputminted{cpp}{../src/math/多项式操作.cpp}

				\subsubsection{更优秀的多项式多点求值}
					这个做法不需要写取模, 求逆也只有一次, 但是神乎其技, 完全搞不懂原理 \\
					清空和复制之类的地方容易抄错, 抄的时候要注意
					\inputminted{cpp}{../src/math/更优秀的多项式多点求值.cpp}

				\subsubsection{多项式快速插值}
					快速插值: 给出$n$个$x_i$与$y_i$, 求一个$n-1$次多项式满足$F(x_i)=y_i$.

考虑拉格朗日插值: 
$$F(x)=\sum_{i=1}^n\frac{\prod_{i\neq j}(x-x_j)}{\prod_{i\neq j}(x_i-x_j)}y_i$$

对每个$i$先求出
$$\prod_{i\neq j}(x_i-x_j)$$

设
$$M(x)=\prod_{i=1}^{n}(x-x_i)$$

那么想要的是
$$\frac{M(x)}{x-x_i}$$

取$x=x_i$时, 上下都为0, 使用洛必达法则, 则原式化为$M'(x)$.

使用分治算出$M(x)$, 使用多点求值算出每个
$$\prod_{i\neq j}(x_i-x_j)=M'(x_i)$$

设
$$\frac{y_i}{\prod_{i\neq j}(x_i-x_j)}=v_i$$

现在要求出$$\sum_{i=1}^{n}v_i\prod_{i\neq j}(x-x_j)$$

使用分治. 设
$$L(x)=\prod_{i=1}^{\lfloor n/2\rfloor}(x-x_i), \; R(x)=\prod_{i=\lfloor n/2\rfloor+1}^n(x-x_i)$$

则原式化为
$$\left( \sum_{i=1}^{\lfloor n/2\rfloor}v_i\prod_{i\neq j,j\leq\lfloor n/2\rfloor}(x-x_j)\right)R(x)+$$
$$\left( \sum_{i=\lfloor n/2\rfloor+1}^{n}v_i\prod_{i\neq j,j>\lfloor n/2\rfloor}(x-x_j)\right)L(x)$$

递归计算, 复杂度$O(n\log^2n)$.

注意由于这里是先计算子问题再合并, 因此不必预处理$(x - x_j)$的乘积, 在分治过程中一起做即可.

				% \subsubsection{多项式复合与复合逆}


				\subsubsection{拉格朗日反演}
					% 用于求复合逆. \\
如果$f(x)$与$g(x)$互为复合逆, 则有 \\
\begin{math}
\begin{aligned}\relax \left[x^n\right]g(x)=\frac{1}{n}\left[x^{n-1}\right]\left(\frac{x}{f(x)}\right)^n\end{aligned} \\
\begin{aligned}\relax \left[x^n\right]h(g(x))=\frac{1}{n}\left[x^{n-1}\right]h'(x)\left(\frac{x}{f(x)}\right)^n\end{aligned}
\end{math}

				\subsubsection{分治FFT}
					\inputminted{cpp}{../src/math/分治FFT.cpp}

				\subsubsection{半在线卷积}
					\inputminted{cpp}{../src/math/半在线卷积.cpp}
				
				\subsubsection{常系数齐次线性递推$O(k\log k\log n)$}
					如果只有一次这个操作可以照抄, 否则就开一个全局flag.
					\inputminted{cpp}{../src/math/常系数齐次线性递推.cpp}

			\subsection{FWT快速沃尔什变换}
				\inputminted{cpp}{../src/math/FWT.cpp}

			\subsection{单纯形}
				\inputminted{cpp}{../src/math/单纯形.cpp}

				\subsubsection{线性规划对偶原理}
					给定一个原始线性规划:

$$
\begin{aligned}
\text{Minimize}&&\sum_{j=1}^n c_j x_j\\
\text{Subject to}&&\sum_{j=1}^n a_{ij} x_j\ge b_i,\\
&&x_j\ge 0
\end{aligned}
$$

定义它的对偶线性规划为:

$$
\begin{aligned}
\text{Maximize}&&\sum_{i=1}^m b_i y_i\\
\text{Subject to}&&\sum_{i=1}^m a_{ij} y_i\le c_j,\\
&&y_i\ge 0
\end{aligned}
$$

用矩阵可以更形象地表示为:
$$
\begin{aligned}
\text{Minimize}&& \ c^T \boldsymbol x &&&& \text{Maximize} && \boldsymbol b^{T}\boldsymbol y\\
\text{Subject to}&& A\boldsymbol x \ge \boldsymbol b, && \Longleftrightarrow && \text{Subject to} && A^T\boldsymbol y \le \boldsymbol c,\\
&& \boldsymbol x\ge 0 &&&&&& \boldsymbol y\ge 0
\end{aligned}
$$


			\subsection{线性代数}
				\subsubsection{矩阵乘法}
					\inputminted{cpp}{../src/math/矩阵乘法.cpp}

				\subsubsection{高斯消元}
					\paragraph{高斯-约当消元法 Gauss-Jordan}
每次选取当前行绝对值最大的数作为代表元,在做浮点数消元时可以很好地保证精度。

\inputminted{cpp}{../src/math/gauss_jordan.cpp}

\paragraph{解线性方程组}
在矩阵的右边加上一列表示系数即可,如果消成上三角的话最后要倒序回代。

\paragraph{求逆矩阵}
维护一个矩阵 $B$,初始设为 $n$ 阶单位矩阵。在消元的同时对 $B$ 进行一样的操作,那么把 $A$ 消成单位矩阵时 $B$ 就是逆矩阵。

\paragraph{行列式}
消成对角之后把代表元乘起来。如果是任意模数,要注意消元时每交换一次行列要取反一次。


				\subsubsection{行列式取模}
					\inputminted{cpp}{../src/math/行列式取模.cpp}

				% \subsubsection{自由元搜索}


				\subsubsection{线性基}
					\inputminted{cpp}{../src/math/线性基.cpp}


				\subsubsection{线性代数知识}
					\paragraph{行列式}
$$ \det A = \sum_{\sigma} \text{sgn}(\sigma) \prod_{i}a_{i, \sigma_i} $$

\paragraph{逆矩阵}
$$ B = A^{-1} \iff AB = 1 $$

\paragraph{代数余子式}
$$ C_{i, j} = (-1) ^ {i + j} M_{i, j} = (-1) ^ {i + j} \left| A ^ {i, j} \right| $$

也就是 $A$ 去掉一行一列之后的行列式。

\paragraph{伴随矩阵}
$$ A^{*} = C^T $$

即代数余子式矩阵的转置。

同时我们有 $$ A^{*} = |A| A^{-1}$$

\paragraph{特征多项式}
$$ P_A(x) = \det \left(Ix - A\right) $$

特征根:特征多项式的所有 $n$ 个根(可能有重根)。

				
				\subsubsection{矩阵树定理}
					

			\subsection{自适应Simpson积分}
				Forked from fstqwq's template.
				\inputminted{cpp}{../src/math/simpson.cpp}

			\subsection{常见数列}
				\subsubsection{斐波那契数}


				\subsubsection{卢卡斯数}


				\subsubsection{伯努利数}
					\begin{math}
    \begin{aligned}B(x)=\sum_{i\ge 0}\frac{B_i x^i}{i!}=\frac x{e^x-1}\end{aligned} \\
    \begin{aligned}B_n=[n=0]-\sum_{i=0}^{n-1}{n\choose i}\frac{B_i}{n-k+1}\end{aligned} \\
    \begin{aligned}\sum_{i=0}^n{n+1\choose i}B_i=0\end{aligned} \\
    \begin{aligned}S_n(m)=\sum_{i=0}^{m-1}i^n=\sum_{i=0}^n{n\choose i}B_{n-i}\frac{m^{i+1}}{i+1}\end{aligned}
\end{math}
				
				\subsubsection{分拆数}
					\inputminted{cpp}{../src/math/分拆数.cpp}
				
				\subsubsection{斯特林数}
					\begin{enumerate}

\item \textbf{第一类斯特林数}

$n\brack k$ 表示 $n$ 个元素划分成 $k$ 个 \textbf{轮换}\ 的方案数。

\paragraph{递推式} ${n \brack k} = {n-1 \brack k-1} + (n-1){n-1 \brack k}$

\paragraph{求同一行} 分治 FFT $O(n\log ^2 n)$,或者倍增 $O(n\log n)$(每次都是 $f(x) = g(x) g(x + d)$ 的形式)。

$$ \begin{aligned} \sum_{k = 0} ^ n {n \brack k} x^k = \prod_{i = 0} ^ {n - 1} (x + i) \end{aligned} $$

\paragraph{求同一列} 用一个轮换的 EGF 做 $k$ 次幂。

$$ \sum_{n = 0} ^ \infty {n \brack k} \frac {x ^ n} {n!} = \frac {\left(\ln (1 - x)\right) ^ k} {k!} = \frac {x ^ k} {k!} \left( \frac {\ln (1 - x)} x \right) ^ k $$

\item \textbf{第二类斯特林数}

$n\brace k$ 表示 $n$ 个元素划分成 $k$ 个子集的方案数。

\paragraph{递推式} ${n \brace k} = {n-1 \brace k-1} + k{n-1 \brace k}$

\paragraph{求某一项} 容斥,狗都会做。

$$ {n \brace k} = \frac 1 {k!} \sum_{i = 0} ^ k (-1) ^ i {k \choose i} (k - i) ^ n = \sum_{i = 0} ^ k \frac {(-1) ^ i} {i!} \frac {(k - i) ^ n} {(k - i)!} $$

\paragraph{求同一行} FFT,狗都会做。

\paragraph{求同一列} EGF:

$$ \sum_{n = 0} ^ \infty {n \brace k} \frac {x ^ n} {n!} = \frac {\left(e ^ x - 1\right) ^ k} {k!} = \frac {x ^ k} {k!} \left( \frac {e ^ x - 1} x \right) ^ k $$

OGF:

$$ \sum_{n = 0} ^ \infty {n \brace k} x ^ n = x ^ k \left(\prod_{i = 1} ^ k (1 - i x)\right) ^ {-1} $$

\item \textbf{斯特林反演}

$$ f(n) = \sum_{k = 0} ^ n {n \brace k} g(k) \iff g(n) = \sum_{k = 0} ^ n (-1) ^ {n - k} {n \brack k} f(k) $$

\item \textbf{幂的转换}

\paragraph{上升幂与普通幂的转换}

$$ x^{\overline{n}}=\sum_{k} {n \brack k} x^k $$

$$ x^n=\sum_{k} {n \brace k} (-1)^{n-k} x^{\overline{k}} $$

\paragraph{下降幂与普通幂的转换}

$$ x^n=\sum_{k} {n \brace k} x^{\underline{k}} = \sum_{k} {x \choose k} {n \brace k} k! $$

$$ x^{\underline{n}}=\sum_{k} {n \brack k} (-1)^{n-k} x^k $$

另外,多项式的 \textbf{点值}\ 表示的每项除以阶乘,卷上 $e^{-x}$ 再乘上阶乘之后是 \textbf{牛顿插值}\ 表示,或者不乘阶乘就是 \textbf{下降幂}\ 系数表示。反过来的转换当然卷上 $e^x$ 就行了。原理是每次差分等价于乘以 $(1 - x)$,展开之后用一次卷积取代多次差分。(参见 \detailedref{NewtonInterpolation}。)

\item \textbf{斯特林多项式(斯特林数关于斜线的性质)}

\paragraph{定义}

$$ \sigma_n(x) = \frac {{x\brack n}} {x(x-1)\dots(x-n)} $$

$\sigma_n(x)$ 的最高次数是 $x^{n - 1}$。(所以作为唯一的特例, $\sigma_0(x) = \frac 1 x$ 不是多项式。)

斯特林多项式实际上非常神奇,它与两类斯特林数都有关系:

$$ {n \brack n-k} = n^{\underline{k+1}} \sigma_k(n) $$

$$ {n \brace n-k} = (-1)^{k+1} n^{\underline{k+1}} \sigma_k(-(n-k)) $$

不过它并不好求。可以 $O(k^2)$ 直接计算前几个点值然后插值,或者如果要推式子的话,可以用后面提到的二阶欧拉数(\detailedref{EulerianNumber})。

\end{enumerate}

				
				\subsubsection{贝尔数}
					$$B_0 = 1,\, B_1 = 1,\, B_2 = 2,\, B_3 = 5,\,$$
$$B_4 = 15,\, B_5 = 52,\, B_6 = 203, \dots$$

$$\begin{aligned}B_n = \sum_{k = 0} ^ n {n\brace k}\end{aligned}$$

\paragraph{递推式}
$$ B_{n + 1} = \sum_{k = 0} ^n {n\choose k} B_k $$

\paragraph{指数生成函数} $B(x) = e^{e^x - 1}$

\paragraph{Touchard 同余} 如果 $p$ 是素数,那么:
$$ B_{n + p^m} \equiv (m B_n + B_{n + 1}) \pmod p $$

				
				\subsubsection{卡特兰数}
					\begin{enumerate}

\item \textbf{卡特兰数}

$$C_n = \frac 1 {n + 1}{2n\choose n} = {2n \choose n} - {2n \choose n - 1}$$

\begin{itemize}
	\item $n$ 个元素按顺序入栈,出栈序列方案数
	\item $n$ 对括号的合法括号序列数
	\item $n + 1$ 个叶子的满二叉树个数
\end{itemize}

\paragraph{递推式}
$$ C_n = \sum_{i = 0} ^ {n - 1} C_i C_{n - i - 1} = C_{n - 1} \frac {4n - 2} {n + 1} $$

\paragraph{普通生成函数} $C(x) = \frac {1 - \sqrt {1 - 4 x}} {2 x}$

\paragraph{扩展} 如果有 $n$ 个 \texttt{(} 和 $m$ 个 \texttt{)},方案数为 ${n + m \choose n} - {n + m \choose m - 1}$。

\item \textbf{施罗德数}

$$ S_n = S_{n-1} + \sum_{i = 0} ^ {n - 1} S_i S_{n - i - 1} $$
$$ (n + 1)s_n = (6n - 3)s_{n - 1} - (n - 2) s_{n - 2} $$

其中 $S_n$ 是(大)施罗德数,$s_n$是小施罗德数(也叫超级卡特兰数)。

除了 $S_0 = s_0 = 1$ 以外,都有 $S_i = 2s_i$。

施罗德数的组合意义:

\begin{itemize}
	\item 从 $(0, 0)$ 走到 $(n, n)$,每次可以走右、上或者右上一步,并且不能超过 $y=x$ 这条线的方案数
	\item 可以有空位,并且括号对数和空位置数加起来等于 $n$ 的合法括号序列数
	\item 凸 $n$ 边形的 \textbf{任意}\ 剖分方案数
\end{itemize}

(有些人会把大(而不是小)施罗德数叫做超级卡特兰数。)

\item \textbf{默慈金数}

$$ M_{n + 1} = M_n + \sum_{i = 0} ^ {n - 1} M_i M_{n - 1 - i} = \frac {(2n + 3)M_n + 3n M_{n - 1}} {n + 3} $$
$$ M_n = \sum_{i = 0} ^ {\frac n 2} {n \choose 2i} C_i $$

\begin{itemize}
	\item 从 $(0, 0)$ 走到 $(n, 0)$,每次可以走右上、右下或者正右方一步,且不能走到 $y<0$ 的位置的方案数
	\item 可以有空位,长为 $n$ 的合法括号序列数
	\item 在圆上的 $n$ 个 \textbf{不同的}\ 点之间画任意条不相交(\textbf{包括端点})的弦的方案数
\end{itemize}

\paragraph{扩展} 默慈金数画的弦不可以共享端点。如果可以共享端点的话是 A054726,参见 \detailedref{oeis}。

\end{enumerate}

			
			\subsection{常用公式及结论}
				\subsubsection{方差}
					$m$ 个数的方差:

$$ s^2 = \frac{\sum_{i=1}^m x_i^2}m - \overline x^2$$

随机变量的方差:$D^2(x)=E(x^2)-E^2(x)$

				
				\subsubsection{连通图计数}
					设大小为$n$的满足一个限制$P$的简单无向图数量为$g_n$, 满足限制$P$且连通的简单无向图数量为$f_n$, 如果已知$g_{1\dots n}$求$f_n$, 可以得到递推式

$$\begin{aligned}f_n=g_n-\sum_{k=1}^{n-1}{n-1\choose k-1}f_k g_{n-k}\end{aligned}$$

这个递推式的意义就是用任意图的数量减掉不连通的数量, 而不连通的数量可以通过枚举$1$号点所在连通块大小来计算.

注意, 由于$f_0=0$, 因此递推式的枚举下界取$0$和$1$都是可以的.

推一推式子会发现得到一个多项式求逆, 再仔细看看, 其实就是一个多项式$\ln$.

				\subsubsection{线性齐次线性常系数递推求通项}
					\includegraphics[scale = 0.265]{../src/math/线性齐次线性常系数递推.png}
				
				\subsubsection{上升幂, 下降幂与普通幂的转换}
					\textbf{上升幂与普通幂的相互转化}

$$ x^{\overline{n}}=\sum_{k} {n \brack k} x^k $$

$$ x^n=\sum_{k} {n \brace k} (-1)^{n-k} x^{\overline{k}} $$

\textbf{下降幂与普通幂的相互转化}

$$ x^n=\sum_{k} {n \brack k} x^{\underline{k}} $$

$$ x^{\underline{n}}=\sum_{k} {n \brace k} (-1)^{n-k} x^k $$

另外, 多项式的点值表示每项除以阶乘之后卷上$e^{-x}$乘上阶乘之后是牛顿插值表示, 或者不乘阶乘就是\textbf{下降幂}系数表示. 反过来的转换当然卷上$e^x$就行了. 原理是每次差分等价于乘以$(1 - x)$, 展开之后用一次卷积取代多次差分.
			
			\subsection{常用生成函数}
				$$\frac x {(1 - x) ^ 2} = \sum_{i \ge 0} i x ^ i$$

$$\frac 1 {(1 - x) ^ k} = \sum_{i \ge 0} {i + k - 1 \choose i} x ^ i = \sum_{i \ge 0} {i + k - 1 \choose k - 1}x^i, \; k > 0$$

$$
\begin{aligned}
	\sum_{i = 0} ^ \infty i^n x^i = \sum_{k = 0} ^ n {n \brace k} k! \frac {x^k} {(1-x) ^ {k + 1}} = \sum_{k = 0} ^ n {n \brace k} k! \frac {x^k (1-x) ^ {n – k}} {(1-x) ^ {n + 1}} \\
	= \frac 1 {(1-x) ^ {n + 1}} \sum_{i = 0} ^ n \frac {x^i} {(n-i)!} \sum_{k = 0} ^ i {n \brace k}k!(n-k)! \frac {(-1)^{i-k}} {(i-k)!}
\end{aligned}
$$

(用上面的方法可以把分子化成一个$n$次以内的多项式, 并且可以用一次卷积求出来.)

如果把$i^n$换成任意的一个$n$次多项式, 那么我们可以求出它的下降幂表示形式(或者说是牛顿插值)的系数$r_i$, 发现用$r_k$替换掉上面的${n \brace k}k!$之后其余过程完全相同.


		\section{数论}

			\subsection{$O(n)$预处理逆元}
				\inputminted{cpp}{../src/numbertheory/O(n)求逆元.cpp}

			\subsection{线性筛}
				\inputminted{cpp}{../src/numbertheory/扩展线性筛.cpp}

			\subsection{杜教筛}
				\inputminted{cpp}{../src/numbertheory/杜教筛.cpp}
			
			\subsection{Powerful Number筛}
				注意 Powerful Number 筛只能求一些 \textbf{特殊}\ 的 \textbf{积性函数}\ 的前缀和。

本质上就是构造一个方便求前缀和的函数,然后做类似杜教筛的操作。

定义 Powerful Number 表示每个质因子幂次都大于 $1$ 的数,显然最多有 $\sqrt n$ 个。

设我们要求和的函数是 $f(n)$,构造一个 \textit{方便求前缀和的}\ \textbf{积性}\ 函数 $g(n)$ 使得 $g(p) = f(p)$。

那么就存在一个积性函数 $h = f * g ^ {-1}$,也就是 $f = g *h$。可以证明 $h(p) = 0$,所以只有 Powerful Number 的 $h$ 值不为 $0$。

$$ _f(i) = \sum_{d = 1} ^ n h(d) S_g \left( \left\lfloor \frac n d \right\rfloor \right) $$

只需要枚举每个 Powerful Number 作为 $d$,然后用杜教筛计算 $g$ 的前缀和即可。

求 $h(d)$ 时要先预处理 $h(p^k)$,显然有

$$ h \left(p ^ k \right) = f \left(p ^ k \right) - \sum_{i = 1} ^ k g \left( p ^ i \right) h \left( p ^ {k - i} \right) $$

处理完之后 DFS 就行了。(显然只需要筛 $\sqrt n$ 以内的质数。)

复杂度取决于杜教筛的复杂度,特殊题目构造的好也可以做到 $O \left( \sqrt n \right)$。

例题:

\begin{itemize}
	\item $f \left( p ^ k \right) = p ^ k \left( p ^ k - 1 \right)$:$g(n) = \text{id}(n) \varphi(n)$。
	\item $f \left( p ^ k \right) = p \, \text{xor} \, k$:$n$ 为偶数时 $g(n) = 3 \varphi(n)$,否则 $g(n) = \varphi(n)$。
\end{itemize}


			\subsection{min25筛}
				问题仍然是计算 \textbf{积性函数}\ $f(n)$ 的前 $n$ 项和。

设 $\sqrt n$ 以内的质数为 $p_1 \dots p_{\pi\left(\sqrt n\right)}$,记

$$ g(n) = \sum_{i = 1} ^ n \left[i \in \mathbb{P}\right] f(i) $$

也就是只考虑质数项的和。

为了方便求出 $g$,构造一个多项式函数 $F(x) = \sum a_i x^i$,满足 $F(p) = f(p)$,这样每个次数就可以分开算贡献。($f(p^c)$ 的形式是无所谓的,只需要能直接求就行。)

再令

$$ h_k(i, n) = \sum_{x = 2} ^ n \left[ x \in \mathbb{P}\ \text{或}\ x\ \text{与前}\ i\ \text{个质数互质} \right] x^k $$

显然 $g(n) = \sum_k a_k h_k\left( \pi\left(\sqrt n\right), n \right)$,递推求出所有 $h_k$ 即可得到 $g$。

考虑 $h$ 的转移,当 $p_i > \sqrt n$ 时显然有 $h_k(i, n) = h_k(i - 1, n)$,否则有

$$ h_k(i, n) =  h_k(i - 1, n) - p_i ^ k h_k\left( i - 1, \left\lfloor \frac n {p_i} \right\rfloor \right) + p_i ^ k \sum_{j = 1} ^ {i - 1} p_j ^ k $$

边界为 $h_k(0, n) = \sum_{i = 2} ^ n i^k$。

求出 $g$ 之后,为了得到所有 $f(i)$ 之和还需要一次递推。设

$$ S(i, n) = \sum_{k = 2} ^ n \left[ k\ \text{与前}\ (i - 1)\ \text{个质数互质} \right] f(k) $$

则

$$ \begin{aligned} S(i, n) = & g(n) - \sum_{k = 1} ^ {i - 1} f(p_k) \\
+ & \sum_{k = i} ^ {p_k \le \sqrt n} \sum_{c = 1} ^ {p_k ^ {c + 1} \le n} \left( S\left( k + 1, \left\lfloor \frac n {p_k ^ c} \right\rfloor \right) f\left( p_k ^ c \right) + f\left( p_k ^ {c + 1} \right) \right) \end{aligned} $$

这里直接递归即可,注意边界应设为 $p_i > n$ 或 $n < 1$ 时 $S(i, n) = 0$。最后的答案即为 $\text{ans} = S(1, n) + f(1)$。

也可以从大到小枚举 $i$ 递推,考虑到只有 $p_i \le \sqrt n$ 时才有递推,可以后缀和优化。不优化的复杂度是 $O(n^{1 - \epsilon})$,优化之后是 $O\left( \frac {n^{\frac 3 4}} {\log n} \right)$,不过一般是不优化更快。

\inputminted{cpp}{../src/numbertheory/min25.cpp}


			\subsection{Miller-Rabin}
				\inputminted{cpp}{../src/numbertheory/Miller-Rabin.cpp}

			\subsection{Pollard's Rho}
				\inputminted{cpp}{../src/numbertheory/Pollard-Rho.cpp}
			
			% \subsection{组合数取模}


			\subsection{扩展欧几里德}
				\inputminted{cpp}{../src/numbertheory/exgcd.cpp}
				\input{../src/numbertheory/exgcd.tex}
			
			\subsection{原根 阶}
				\textbf{阶}: 最小的整数$k$使得$a ^ k \equiv 1 \pmod p$, 记为$\delta_p(a)$.

显然$a$在原根以下的幂次是两两不同的.

一个性质: 如果$a, b$均与$p$互质, 则 $ \delta_p(ab)=\delta_p(a)\delta_p(b) $ 的充分必要条件是$ \gcd\big(\delta_p(a),\delta_p(b)\big)=1 $.

另外, 如果$a$与$p$互质, 则有$ \delta_p(a^k)=\dfrac{\delta_p(a)}{\gcd\big(\delta_p(a),k\big)} $. (也就是环上一次跳$k$步的周期.)

\textbf{原根}: 阶等于$\varphi(p)$的数.

只有形如$2, 4, p ^ k, 2 p ^ k$($p$是奇素数)的数才有原根, 并且如果一个数$n$有原根, 那么原根的个数是$\varphi(\varphi(n))$个.

暴力找原根代码:
\begin{minted}{python}
def split(n): # 分解质因数
    i = 2
    a = []
    while i * i <= n:
        if n % i == 0:
            a.append(i)

            while n % i == 0:
                n /= i

        i += 1

    if n > 1:
        a.append(n)

    return a
    
def getg(p): # 找原根
    def judge(g):
        for i in d:
            if pow(g, (p - 1) / i, p) == 1:
                return False
        return True

    d = split(p - 1)
    g = 2

    while not judge(g):
        g += 1

    return g

print(getg(int(input())))
\end{minted}
			
			\subsection{常用公式}
				\subsubsection{莫比乌斯反演}
	$$ \begin{aligned}
		f(n) = \sum_{d | n} g(d) \Leftrightarrow g(n) = \sum_{d | n} \mu\left( \frac n d \right) f(d) \\
		f(d) = \sum_{d | k} g(k) \Leftrightarrow g(d) = \sum_{d | k} \mu\left( \frac k d \right) f(k)
	\end{aligned} $$

\subsubsection{其他常用公式}
	$$\mu * I = e \quad (e(n) = [n = 1])$$

	$$\varphi * I = id $$

	$$\mu * id = \varphi $$

	$$\sigma_0 = I * I ,\, \sigma_1 = id * I ,\, \sigma_k = id^{k - 1} * I$$

	$$\sum_{i = 1} ^ n \left[(i, n) = 1\right] i = n \frac {\varphi(n) + e(n)} 2$$
	
	$$\sum_{i = 1} ^ n \sum_{j = 1} ^ i \left[(i, j) = d\right] = S_\varphi \left( \left\lfloor \frac n d \right\rfloor \right)$$

	$$\sum_{i = 1} ^ n \sum_{j = 1} ^ m \left[(i, j) = d\right] = \sum_{d | k} \mu\left( \frac k d \right) \left\lfloor \frac n k \right\rfloor \left\lfloor \frac m k \right\rfloor$$

	$$ \sum_{i = 1} ^ n f(i) \sum_{j = 1} ^ {\left\lfloor \frac n i \right\rfloor} g(j) = \sum_{i = 1} ^ n g(i) \sum_{j = 1} ^ {\left\lfloor \frac n i \right\rfloor} f(j) $$
				

		\section{图论}

			\subsection{最小生成树}
				\subsubsection{Boruvka算法}
					思想:每次选择连接每个连通块的最小边,把连通块缩起来。

每次连通块个数至少减半,所以迭代 $O(\log n)$ 次即可得到最小生成树。

一种比较简单的实现方法:每次迭代遍历所有边,用并查集维护连通性和每个连通块的最小边权。

应用:最小异或生成树

				
				\subsubsection{动态最小生成树}
					\inputminted{cpp}{../src/graph/动态最小生成树.cpp}
				
				\subsubsection{最小树形图(朱刘算法)}
					对每个点找出最小的入边, 如果是一个DAG那么就已经结束了.

否则把环都缩起来再跑一遍, 直到没有环为止.

可以用可并堆优化到$O(m\log n)$, 需要写一个带懒标记的左偏树.
				
				\subsubsection{Steiner Tree 斯坦纳树}
					{\bfseries 问题}: 一张图上有$k$个关键点, 求让关键点两两连通的最小生成树

{\bfseries 做法}: 状压DP, $f_{i,S}$表示以$i$号点为树根, $i$与$S$中的点连通的最小边权和

转移有两种:
\begin{enumerate}
	\item 枚举子集: $$\begin{aligned}f_{i, S} = \min_{T\subset S} \left\{f_{i, T} + f_{i, S\setminus T}\right\}\end{aligned}$$
	\item 新加一条边: $$\begin{aligned}f_{i, S} = \min_{(i, j)\in E} \left\{f_{j, S} + w_{i, j}\right\}\end{aligned}$$
\end{enumerate}
第一种直接枚举子集DP就行了, 第二种可以用SPFA或者Dijkstra松弛(显然负边一开始全选就行了, 所以只需要处理非负边).

复杂度$O(n 3^k + 2^k m\log n)$.
			
			
			\subsection{最短路}
				\subsubsection{Dijkstra}
					见k短路(注意那边是求到$t$的最短路)
				
				\subsubsection{Johnson算法(负权图多源最短路)}
					首先前提是图没有负环.

先任选一个起点$s$, 跑一边SPFA, 计算每个点的势$h_u = d_{s, u}$, 然后将每条边$u\to v$的权值$w$修改为$w + h[u] - h[v]$即可, 由最短路的性质显然修改后边权非负.

然后对每个起点跑Dijkstra, 再修正距离$d_{u, v} = d'_{u, v} - h_u + h_v$即可, 复杂度$O(nm\log n)$, 在稀疏图上是要优于Floyd的.
				
				% \subsubsection{差分约束}


				\subsubsection{k短路}
					\inputminted{cpp}{../src/graph/k短路.cpp}
			
			\subsection{Tarjan算法}
				\subsubsection{强连通分量}
					\inputminted{cpp}{../src/graph/强连通分量.cpp}
				
				\subsubsection{割点 点双}
					\inputminted{cpp}{../src/graph/割点点双.cpp}

				\subsubsection{桥 边双}


			\subsection{仙人掌}
				一般来说仙人掌问题都可以通过圆方树转成有两种点的树上问题来做.
				\subsubsection{仙人掌DP}
					\inputminted{cpp}{../src/graph/仙人掌DP.cpp}
			
			\subsection{二分图}
				\subsubsection{匈牙利}
					\inputminted{cpp}{../src/graph/hungary.cpp}

				% \subsubsection{Hopcroft-Karp}


				\subsubsection{KM二分图最大权匹配}
					\inputminted{cpp}{../src/graph/KM二分图最大权匹配.cpp}

				\subsubsection{二分图原理}
					\textbf{最大匹配的可行边与必须边}

\begin{itemize}
	\item 可行边: 一条边的两个端点在残量网络中处于同一个SCC, 不论是正向边还是反向边.

	\item 必须边: 一条属于当前最大匹配的边, 且残量网络中两个端点不在同一个SCC中.
\end{itemize}

\textbf{独立集}

二分图独立集可以看成最小割问题, 割掉最少的点使得S和T不连通, 则剩下的点自然都在独立集中.

所以独立集输出方案就是求出不在最小割中的点, 独立集的必须点/可行点就是最小割的不可行点/非必须点.

割点等价于割掉它与源点或汇点相连的边, 可以通过设置中间的边权为无穷以保证不能割掉中间的边, 然后按照上面的方法判断即可.

(由于一个点最多流出一个流量, 所以中间的边权其实是可以任取的.)
			
			\subsection{一般图匹配}
				\subsubsection{高斯消元}
					\inputminted{cpp}{../src/graph/基于线性代数的一般图匹配.cpp}

				\subsubsection{带花树}
					\inputminted{cpp}{../src/graph/带花树.cpp}
				
				\subsubsection{带权带花树}
					(有一说一这玩意实在太难写了, 抄之前建议先想想算法是不是假的或者有SB做法)
					\inputminted{cpp}{../src/graph/带权带花树.cpp}
				
				\subsubsection{原理}
					设图$G$的Tutte矩阵是$\tilde A$, 首先是最基础的引理:

\begin{itemize}
	\item $G$的最大匹配大小是$\frac 1 2 \text{rank}{\tilde A}$.
	
	\item $({\tilde A} ^{-1}) _{i, j} \ne 0$当且仅当$G-\{v_i, v_j\}$有完美匹配.
		\subitem (考虑到逆矩阵与伴随矩阵的关系, 这是显然的.)
\end{itemize}

构造最大匹配的方法见板子. 对于更一般的问题, 可以借助构造方法转化为完美匹配问题.

设最大匹配的大小为$k$, 新建$n - 2 k$个辅助点, 让它们和其他所有点连边, 那么如果一个点匹配了一个辅助点, 就说明它在原图的匹配中不匹配任何点.

\begin{itemize}
	\item 最大匹配的可行边: 对原图中的任意一条边$(u, v)$, 如果删掉$u, v$后新图仍然有完美匹配(也就是${\tilde A} ^ {-1}_{u, v} \ne 0$), 则它是一条可行边.
	
	\item 最大匹配的必须边: \textbf{待补充}
	
	\item 最大匹配的必须点: 可以删掉这个点和一个辅助点, 然后判断剩下的图是否还有完美匹配, 如果有则说明它不是必须的, 否则是必须的. 只需要用到逆矩阵即可.
	
	\item 最大匹配的可行点: 显然对于任意一个点, 只要它不是孤立点, 就是可行点.
\end{itemize}

			
			% \subsection{支配树}
			

			\subsection{2-SAT}
				如果限制满足对称性, 那么可以使用Tarjan算法求SCC搞定. 
				
				具体来说就是, 如果某个变量的两个点在同一SCC中则显然无解, 否则按拓扑序倒序尝试选择每个SCC即可.

				如果要字典序最小或者不满足对称性就用dfs, 注意可以压位优化.


			\subsection{最大流}
				\subsubsection{Dinic}
					\inputminted{cpp}{../src/graph/Dinic.cpp}

				\subsubsection{ISAP}
					\textbf{可能有毒, 慎用.}
					\inputminted{cpp}{../src/graph/ISAP.cpp}

				\subsubsection{HLPP最高标号预流推进}
					\inputminted{cpp}{../src/graph/HLPP.cpp}
			
			\subsection{费用流}
				\subsubsection{SPFA费用流}
					\inputminted{cpp}{../src/graph/SPFA费用流.cpp}

				\subsubsection{Dijkstra费用流}
					也叫原始-对偶费用流。

原理和求多源最短路的 Johnson 算法是一样的,都是给每个点维护一个势 $h_u$,使得对任何有向边 $u\to v$ 都满足 $w + h_u - h_v \ge 0$。

如果有负费用则从 $s$ 开始跑一遍 SPFA 初始化,否则可以直接初始化 $h_u = 0$。

每次增广时得到的路径长度就是 $d_{s, t} + h_t$,增广之后让所有 $h_u = h'_u + d'_{s, u}$,直到 $d_{s, t} = +\infty$(最小费用最大流)或 $d_{s, t} \ge 0$(最小费用流)为止。

注意最大费用流要转成取负之后的最小费用流,因为 Dijkstra 求的是最短路。

\inputminted{cpp}{../src/graph/dijkstra费用流.cpp}


				% \subsubsection{zkw费用流}


				% \subsubsection{网络单纯形}
			

			\subsection{网络流原理}
				\subsubsection{最大流}

\textbf{判断一条边是否必定满流}

在残量网络中跑一遍Tarjan, 如果某条满流边的两端处于同一SCC中则说明它不一定满流. (因为可以找出包含反向边的环, 增广之后就不满流了.)

\subsubsection{最小割}

\textbf{最小割输出一种方案}

在残量网络上从$S$开始floodfill, 源点可达的记为$S$集, 不可达的记为$T$, 如果一条边的起点在$S$集而终点在$T$集, 就将其加入最小割中.

\textbf{最小割的可行边与必须边}

\begin{itemize}
	\item 可行边: 满流, 且残量网络上不存在$S$到$T$的路径, 也就是$S$和$T$不在同一SCC中. (实际上也就是最大流必定满流的边.)

	\item 必须边: 满流, 且残量网络上S可达起点, 终点可达T.
\end{itemize}

\textbf{字典序最小的最小割}

\subsubsection{费用流}


\subsubsection{上下界网络流}

\textbf{有源汇上下界最大流}

新建超级源汇$S', T'$, 然后如图所示转化每一条边.

\includegraphics[scale = 0.5]{../src/graph/上下界网络流.png}

然后从$S'$到$S$, 从$T$到$T'$分别连容量为正无穷的边即可.

\textbf{有源汇上下界最小流}

按照上面的方法转换后先跑一遍最大流, 然后撤掉超级源汇, 反过来跑一次最大流退流, 最大流减去退掉的流量就是最小流.

\textbf{无源汇上下界可行流}

转化方法和上面的图是一样的, 只不过不需要考虑原有的源汇了.

在新图跑一遍最大流之后检查一遍辅助边, 如果有辅助边没满流则无解, 否则把每条边的流量加上$b$就是一组可行方案.

\subsubsection{常见建图方法}


\subsubsection{例题}


			\subsection{弦图相关}
				From NEW CODE!!
				\input{../src/graph/弦图相关.tex}

			% \subsection{弦图}
				% \subsubsection{完美消除序列, 最大势算法}


				% \subsubsection{性质}
			

			% \subsection{其他}
				% \subsubsection{Stoer-Wagner全局最小割}


				% \subsubsection{最小割树}


				% \subsubsection{最大团搜索}
			

		\section{数据结构}

			\subsection{线段树}
				\subsubsection{非递归线段树}
					\sout{让 \textit{fstqwq} 手撕}\ 我队友呢?

\begin{itemize}
	\item 如果 $M = 2^k$,则只能维护 $[1, M - 2]$ 范围
	\item 找叶子:$i$ 对应的叶子就是 $i + M$
	\item 单点修改:找到叶子然后向上跳
	\item 区间查询:左右区间各扩展一位,转换成开区间查询
	\inputminted{cpp}{../src/datastructure/非递归线段树单点修改.cpp}

\end{itemize}
	
区间修改要标记永久化,并且求区间和和求最值的代码不太一样。

\textbf{区间加,区间求和}
\inputminted{cpp}{../src/datastructure/非递归线段树区间加区间求和.cpp}

\textbf{区间加,区间求最大值}
\inputminted{cpp}{../src/datastructure/非递归线段树区间加区间求最大值.cpp}

				
				\subsubsection{线段树维护矩形并}
					为线段树的每个结点维护$cover_i$表示这个区间被完全覆盖的次数.

					更新时分情况讨论, 如果当前区间已被完全覆盖则长度就是区间长度, 否则长度是左右儿子相加.
					\inputminted{cpp}{../src/datastructure/线段树维护矩形并.cpp}

				\subsubsection{主席树}
					这种东西能不能手撕啊
	
			\subsection{陈丹琦分治}
				\inputminted{cpp}{../src/datastructure/CDQ分治.cpp}
	
			\subsection{整体二分}
				修改和询问都要划分, 备份一下, 递归之前copy回去.

				如果是满足可减性的问题(例如查询区间$k$小数)可以直接在划分的时候把询问的$k$修改一下. 否则需要维护一个全局的数据结构, 一般来说可以先递归右边再递归左边, 具体维护方法视情况而定.

			% \subsection{块状链表}
	
	
			\subsection{平衡树}
				pb\_ds平衡树在misc(倒数第二章)里.

				\subsubsection{Treap}
					\inputminted{cpp}{../src/datastructure/Treap.cpp}
					
				\subsubsection{无旋Treap/可持久化Treap}
					\inputminted{cpp}{../src/datastructure/无旋Treap.cpp}
		
		
				\subsubsection{Splay}
					如果插入的话可以直接找到底然后splay一下, 也可以直接splay前驱后继.
					\inputminted[]{cpp}{../src/datastructure/文艺平衡树.cpp}
				
				
			\subsection{树分治}
				% \subsubsection{静态树分治}

				
				\subsubsection{动态树分治}
					\inputminted{cpp}{../src/datastructure/动态树分治.cpp}

				\subsubsection{紫荆花之恋}
					\inputminted{cpp}{../src/datastructure/紫荆花之恋.cpp}
	
			\subsection{LCT}
				\subsubsection{不换根(弹飞绵羊)}
					\inputminted{cpp}{../src/datastructure/LCT(不换根).cpp}
			
				\subsubsection{换根/维护生成树}
					\inputminted{cpp}{../src/datastructure/LCT(换根).cpp}

				\subsubsection{维护子树信息}
					\inputminted{cpp}{../src/datastructure/LCT维护子树信息.cpp}
					
				\subsubsection{模板题:动态QTREE4(询问树上相距最远点)}
						\inputminted{cpp}{../src/datastructure/动态QTREE4.cpp}
	
			\subsection{K-D树}

				
				\subsubsection{动态K-D树}
					\inputminted{cpp}{../src/datastructure/动态KD树.cpp}
	
			% \subsection{树分块}
	
	
			% \subsection{树上莫队}
			
	
			\subsection{虚树}
				\inputminted{cpp}{../src/datastructure/虚树.cpp}
	
			\subsection{长链剖分}
				\inputminted{cpp}{../src/datastructure/长链剖分.cpp}
	
			\subsubsection{梯子剖分}
				\inputminted{cpp}{../src/datastructure/梯子剖分.cpp}
					
			\subsection{左偏树}
				(参见k短路)
	
			\subsection{常见根号思路}
				
\noindent{
	{\large\bfseries{通用}}
	\begin{itemize}
		\item 出现次数大于$\sqrt n$的数不会超过$\sqrt n$个
		\item 对于带修改问题, 如果不方便分治或者二进制分组, 可以考虑对操作分块, 每次查询时暴力最后的$\sqrt n$个修改并更正答案
		\item {\bfseries 根号分治}: 如果分治时每个子问题需要$O(N)$(N是全局问题的大小)的时间, 而规模较小的子问题可以$O(n^2)$解决, 则可以使用根号分治
			\begin{itemize}
				\item 规模大于$\sqrt n$的子问题用$O(N)$的方法解决, 规模小于$\sqrt n$的子问题用$O(n^2)$暴力
				\item 规模大于$\sqrt n$的子问题最多只有$\sqrt n$个
				\item 规模不大于$\sqrt n$的子问题大小的平方和也必定不会超过$n\sqrt n$
			\end{itemize}
		\item 如果输入规模之和不大于$n$(例如给定多个小字符串与大字符串进行询问), 那么规模超过$\sqrt n$的问题最多只有$\sqrt n$个
	\end{itemize}
}

\noindent{
	{\large\bfseries{序列}}
	\begin{itemize}
		\item 某些维护序列的问题可以用分块/块状链表维护
		\item 对于静态区间询问问题, 如果可以快速将左/右端点移动一位, 可以考虑莫队
			\begin{itemize}
				\item 如果强制在线可以分块预处理, 但是一般空间需要$n\sqrt n$ 
					\begin{itemize}
						\item 例题: 询问区间中有几种数出现次数恰好为$k$, 强制在线
					\end{itemize}
				\item 如果带修改可以试着想一想带修莫队, 但是复杂度高达$n^{\frac 5 3}$
			\end{itemize}
		\item 线段树可以解决的问题也可以用分块来做到$O(1)$询问或是$O(1)$修改, 具体要看哪种操作更多
	\end{itemize}
}

\noindent{
	{\large\bfseries{树}}
	\begin{itemize}
		\item 与序列类似, 树上也有树分块和树上莫队
			\begin{itemize}
				\item 树上带修莫队很麻烦, 常数也大, 最好不要先考虑
				\item 树分块不要想当然
			\end{itemize}
		\item 树分治也可以套根号分治, 道理是一样的
	\end{itemize}
}

\noindent{
	{\large\bfseries{字符串}}
	\begin{itemize}
		\item 循环节长度大于$\sqrt n$的子串最多只有$O(n)$个, 如果是极长子串则只有$O(\sqrt n)$个
	\end{itemize}
}

\noindent {
	{\large\bfseries{关于莫队}}

	莫队是可以改造成只有插入和撤销(或者只有删除和撤销)的版本的.

	例如维护dfs序时就可以使用链表, 配合只有删除的莫队就可以做到$O(n\sqrt n)$.

	另外如果$n$和$q$不平衡, 块大小应该设为$\frac n {\sqrt q}$.
}

		\section{字符串}
			\subsection{KMP}
				\inputminted{cpp}{../src/string/KMP.cpp}
				
				\subsubsection{ex-KMP}
					\inputminted{cpp}{../src/string/exKMP.cpp}

			\subsection{AC自动机}
				\inputminted{cpp}{../src/string/AC自动机.cpp}

			\subsection{后缀数组}
				% \subsubsection{倍增}
					

				\subsubsection{SA-IS}
					\inputminted{cpp}{../src/string/sais.cpp}
			
				\subsubsection{SAMSA}
				 	\inputminted{cpp}{../src/string/SAMSA.cpp}

			\subsection{后缀平衡树}
				如果不需要查询排名,只需要维护前驱后继关系的题目,可以直接用二分哈希 + \texttt{set} 去做。

一般的题目需要查询排名,这时候就需要写替罪羊树或者 Treap 维护 \texttt{tag}。插入后缀时,如果首字母相同,只需比较各自删除首字母后的 \texttt{tag} 大小即可。

(Treap 也具有重量平衡树的性质,每次插入后影响到的子树大小期望是 $O(\log n)$ 的,所以每次做完插入操作之后直接暴力重构子树内tag就行了。)


			\subsection{后缀自动机}
				(广义后缀自动机复杂度就是$O\left(n\left|\Sigma\right|\right)$, 也没法做到更低了)
				\inputminted{cpp}{../src/string/后缀自动机.cpp}

			\subsection{回文树}
				\inputminted{cpp}{../src/string/回文树.cpp}

				\subsubsection{广义回文树}
					(代码是梯子剖分的版本, 压力不大的题目换成直接倍增就好了,常数只差不到一倍)
					\inputminted{cpp}{../src/string/广义回文树.cpp}

			% \subsection{序列自动机}


			\subsection{Manacher马拉车}
				\inputminted{cpp}{../src/string/manacher.cpp}
			
			% \subsection{All Substring LCS}


			\subsection{字符串原理}
				KMP和AC自动机的fail指针存储的都是它在串或者字典树上的最长后缀, 因此要判断两个前缀是否互为后缀时可以直接用fail指针判断. 当然它不能做子串问题, 也不能做最长公共后缀.

后缀数组利用的主要是LCP长度可以按照字典序做RMQ的性质, 与某个串的LCP长度$\ge$某个值的后缀形成一个区间. 另外一个比较好用的性质是本质不同的子串个数 = 所有子串数 - 字典序相邻的串的height.

后缀自动机实际上可以接受的是所有后缀, 如果把中间状态也算上的话就是所有子串. 它的fail指针代表的也是当前串的后缀, 不过注意每个状态可以代表很多状态, 只要右端点在right集合中且长度处在$(val_{par_p}, val_p]$中的串都被它代表.

后缀自动机的fail树也就是\textbf{反串}的后缀树. 每个结点代表的串和后缀自动机同理, 两个串的LCP长度也就是他们在后缀树上的LCA.

			\section{动态规划}
				\subsection{决策单调性$O(n\log n)$}
					\inputminted{cpp}{../src/DP/决策单调性.cpp}
				
				\subsection{例题}

				
			\section{Miscellaneous}
				\subsection{$O(1)$快速乘}
					\inputminted{cpp}{../src/misc/O(1)快速乘.cpp}
				
				\subsection{Python Decimal}
					\inputminted{python}{../src/misc/decimal.py}
				
				\subsection{$O(n^2)$高精度}
					\inputminted{cpp}{../src/misc/高精度.cpp}
				
				\subsection{笛卡尔树}
					\inputminted{cpp}{../src/misc/笛卡尔树.cpp}
				
				\subsection{常用NTT素数及原根}
					\begin{tabular}{|c|c|c|c|}
	\hline $p = r \times 2 ^ k + 1$ &  $r$  & $k$  & 最小原根 \\
	\hline $104857601$              &  $25$ & $22$ & $3$ \\
	\hline $167772161$              &  $5$  & $25$ & $3$ \\
     \hline $469762049$      &  $7$   &  $26$  &    $3$ \\
     \hline $\mathbf{985661441}$      & $235$  &  $22$  &    $3$ \\
  \hline $\mathbf{998244353}$    & $119$  &  $23$  &    $3$ \\
  \hline $1004535809$    & $479$  &  $21$  &    $3$ \\
  \hline $1005060097 ^ *$      & $1917$ &  $19$  &  $\emph{5}$ \\
  \hline $\emph{2013265921}$    &  $15$  &  $27$  &  $\emph{31}$ \\
    \hline $2281701377$      &  $17$  &  $27$  &    $3$ \\
 \hline $31525197391593473$  &  $7$   &  $52$  &    $3$ \\
\hline $180143985094819841$  &  $5$   &  $55$  &  $\emph{6}$ \\
\hline $1945555039024054273$ &  $27$  &  $56$  &  $\emph{5}$ \\
\hline $4179340454199820289$ &  $29$  &  $57$  &    $3$ \\
\hline
\end{tabular}

*注: $1005060097$有点危险, 在变化长度大于$524288 = 2 ^ {19}$时不可用.

				\subsection{xorshift}
					\inputminted{cpp}{../src/misc/xorshift.cpp}
				
				\subsection{枚举子集}
					(注意这是$t\ne 0$的写法, 如果可以等于$0$需要在循环里手动\mintinline{cpp}{break})
\begin{minted}{cpp}
for (int t = s; t; (--t) &= s) {
    // do something
}
\end{minted}
					
				\subsection{STL}
					\noindent

\begin{enumerate}
	\item vector
		\begin{itemize}
			\item \mintinline{cpp}{vector(int nSize)}:创建一个vector,元素个数为nSize
			\item \mintinline{cpp}{vector(int nSize,const t& t)}:创建一个vector,元素个数为nSize,且值均为t
			\item \mintinline{cpp}{vector(begin,end)}:复制[begin,end)区间内另一个数组的元素到vector中
			\item \mintinline{cpp}{void assign(int n,const T& x)}:设置向量中前n个元素的值为x
			\item \mintinline{cpp}{void assign(const_iterator first,const_iterator last)}:向量中[first,last)中元素设置成当前向量元素
		\end{itemize}
	
	\item list
		\begin{itemize}
			\item \mintinline{cpp}{assign()} 给list赋值 
			\item \mintinline{cpp}{back()} 返回最后一个元素 
			\item \mintinline{cpp}{begin()} 返回指向第一个元素的迭代器 
			\item \mintinline{cpp}{clear()} 删除所有元素 
			\item \mintinline{cpp}{empty()} 如果list是空的则返回true 
			\item \mintinline{cpp}{end()} 返回末尾的迭代器
			\item \mintinline{cpp}{erase()} 删除一个元素
			\item \mintinline{cpp}{front()} 返回第一个元素
			\item \mintinline{cpp}{insert()} 插入一个元素到list中
			\item \mintinline{cpp}{max_size()} 返回list能容纳的最大元素数量
			\item \mintinline{cpp}{merge()} 合并两个list
			\item \mintinline{cpp}{pop_back()} 删除最后一个元素
			\item \mintinline{cpp}{pop_front()} 删除第一个元素
			\item \mintinline{cpp}{push_back()} 在list的末尾添加一个元素
			\item \mintinline{cpp}{push_front()} 在list的头部添加一个元素
			\item \mintinline{cpp}{rbegin()} 返回指向第一个元素的逆向迭代器
			\item \mintinline{cpp}{remove()} 从list删除元素
			\item \mintinline{cpp}{remove_if()} 按指定条件删除元素
			\item \mintinline{cpp}{rend()} 指向list末尾的逆向迭代器
			\item \mintinline{cpp}{resize()} 改变list的大小
			\item \mintinline{cpp}{reverse()} 把list的元素倒转
			\item \mintinline{cpp}{size()} 返回list中的元素个数
			\item \mintinline{cpp}{sort()} 给list排序
			\item \mintinline{cpp}{splice()} 合并两个list
			\item \mintinline{cpp}{swap()} 交换两个list
			\item \mintinline{cpp}{unique()} 删除list中重复的元
		\end{itemize}
\end{enumerate}


				\subsection{pb\_ds}
					\subsubsection{哈希表}

\begin{minted}{cpp}
#include<ext/pb_ds/assoc_container.hpp>
#include<ext/pb_ds/hash_policy.hpp>
using namespace __gnu_pbds;

cc_hash_table<string, int> mp1; // 拉链法
gp_hash_table<string, int> mp2; // 查探法(快一些)
\end{minted}

\subsubsection{堆}

默认也是大根堆, 和\mintinline{cpp}{std::priority_queue}保持一致.

\begin{minted}{cpp}
#include<ext/pb_ds/priority_queue.hpp>
using namespace __gnu_pbds;

__gnu_pbds::priority_queue<int> q;
__gnu_pbds::priority_queue<int, greater<int>, pairing_heap_tag> pq;
\end{minted}

效率参考:

\includegraphics[scale = 0.385]{../src/misc/pbds_heap.png}

常用操作:

\begin{itemize}

	\item \mintinline{cpp}{push()}: 向堆中压入一个元素, 返回迭代器
	\item \mintinline{cpp}{pop()}: 将堆顶元素弹出
	\item \mintinline{cpp}{top()}: 返回堆顶元素
	\item \mintinline{cpp}{size()}: 返回元素个数
	\item \mintinline{cpp}{empty()}: 返回是否非空
	\item \mintinline{cpp}{modify(point_iterator, const key)}: 把迭代器位置的 \mintinline{cpp}{key} 修改为传入的 \mintinline{cpp}{key}
	\item \mintinline{cpp}{erase(point_iterator)}: 把迭代器位置的键值从堆中删除
	\item \mintinline{cpp}{join(__gnu_pbds::priority_queue &other)}: 把 \mintinline{cpp}{other} 合并到 \mintinline{cpp}{*this}, 并把 \mintinline{cpp}{other} 清空
\end{itemize}

\subsubsection{平衡树}

\begin{minted}{cpp}
#include <ext/pb_ds/tree_policy.hpp>
#include <ext/pb_ds/assoc_container.hpp>
using namespace __gnu_pbds;

tree<int, null_type, less<int>, rb_tree_tag, tree_order_statistics_node_update> t;

// rb_tree_tag 红黑树(还有splay_tree_tag和ov_tree_tag, 后者不知道是什么)
\end{minted}

注意第五个参数要填\mintinline{cpp}{tree_order_statistics_node_update}才能使用排名操作.

\begin{itemize}
	\item \mintinline{cpp}{insert(x)}: 向树中插入一个元素x, 返回\mintinline{cpp}{pair<point_iterator, bool>}
	\item \mintinline{cpp}{erase(x)}: 从树中删除一个元素/迭代器x, 返回一个 \mintinline{cpp}{bool} 表明是否删除成功
	\item \mintinline{cpp}{order_of_key(x)}: 返回x的排名, 0-based
	\item \mintinline{cpp}{find_by_order(x)}: 返回排名(0-based)所对应元素的迭代器
	\item \mintinline{cpp}{lower_bound(x) / upper_bound(x)}: 返回第一个$\ge$或者>x的元素的迭代器
	\item \mintinline{cpp}{join(x)}: 将x树并入当前树, 前提是两棵树的类型一样, 并且二者值域不能重叠, x树会被删除
	\item \mintinline{cpp}{split(x,b)}: 分裂成两部分, 小于等于x的属于当前树, 其余的属于b树
	\item \mintinline{cpp}{empty()}: 返回是否为空
	\item \mintinline{cpp}{size()}: 返回大小
\end{itemize}

(注意平衡树不支持多重值, 如果需要多重值, 可以再开一个\mintinline{cpp}{unordered_map}来记录值出现的次数, 将\mintinline{cpp}{x<<32}后加上出现的次数后插入. 注意此时应该为long long类型.)

				\subsection{rope}
					\inputminted{cpp}{../src/misc/rope.cpp}
				
				\subsection{编译选项}
					\begin{itemize}
	\item \texttt{-O2 -g -std=c++20}:狗都知道
	\item \texttt{-Wall -Wextra -Wshadow -Wconversion}:更多警告
		\subitem --\; \texttt{-Werror}:强制将所有 Warning 变成 Error
	\item \texttt{-fsanitize=(address/undefined/ftrapv)}:检查数组越界/有符号整数溢出(算 UB)
		\subitem --\; 调试神器,在遇到错误时会输出信息。
		\subitem --\; 注意无符号类型溢出不算 UB。
	% \item \mintinline{cpp}{-fno-ms-extensions}:关闭一些和 msvc 保持一致的特性,例如不标返回值类型的函数会报 CE 而不是默认为 \mintinline{cpp}{int}。
	% 	\subitem --\;但是不写 \texttt{return} 的话它还是管不了。
	\item \mintinline{cpp}{#define debug(x) cout << #x << " = " << x << endl}
\end{itemize}

					
				\subsection{注意事项}

					\subsubsection{常见下毒手法}
						\noindent
\begin{itemize}
    \item 0/1base是不是搞混了
    \item 高精度高低位搞反了吗
    \item 线性筛抄对了吗
    \item 快速乘抄对了吗
    \item \mintinline{cpp}{i <= n, j <= m}
    \item sort比较函数是不是比了个寂寞
    \item 该取模的地方都取模了吗
    \item 边界情况(+1-1之类的)有没有想清楚
    \item \bfseries{特判是否有必要, 确定写对了吗}
\end{itemize}

					\subsubsection{场外相关}
						\input{../src/attention/场外相关.tex}

					\subsubsection{做题策略与心态调节}
						\noindent
\begin{itemize}
	\item 拿到题后立刻按照商量好的顺序读题, 前半小时最好跳过题意太复杂的题(除非被过穿了)
	
	\item 签到题写完不要激动, 稍微检查一下最可能的下毒点再交, 避免无谓的罚时
		\subitem 一两行的那种傻逼题就算了
		
	\item 读完题及时输出题意, 一方面避免重复读题, 一方面也可以让队友有一个初步印象, 方便之后决定开题顺序
	
	\item 如果不能确定题意就不要贸然输出甚至上机,  尤其是签到题,  因为样例一般都很弱
	
	\item 一个题如果卡了很久又有其他题可以写, 那不妨先放掉写更容易的题, 不要在一棵树上吊死
		\subitem 不要被一两道题搞得心态爆炸, 一方面急也没有意义, 一方面你很可能真的离AC就差一步
		
	\item 榜是不会骗人的, 一个题如果被不少人过了就说明这个题很可能并没有那么难;如果不是有十足的把握就不要轻易开没什么人交的题;另外不要忘记最后一小时会封榜
	
	\item 想不出题/找不出毒自然容易犯困, 一定不要放任自己昏昏欲睡, 最好去洗手间冷静一下, 没有条件就站起来踱步
	
	\item 思考的时候不要挂机, 一定要在草稿纸上画一画, 最好说出声来最不容易断掉思路
	
	\item 出完算法一定要check一下样例和一些trivial的情况, 不然容易写了半天发现写了个假算法
	
	\item 上机前有时间就提前给需要思考怎么写的地方打草稿, 不要浪费机时
	
	\item 查毒时如果最难的地方反复check也没有问题, 就从头到脚仔仔细细查一遍, 不要放过任何细节, 即使是并查集和sort这种东西也不能想当然
	
	\item 后半场如果时间不充裕就不要冒险开难题, 除非真的无事可做
		\subitem 如果是没写过的东西也不要轻举妄动, 在有其他好写的题的时候就等一会再说
		
	\item 大多数时候都要听队长安排, 虽然不一定最正确但可以保持组织性
	
	% \item 最好注意一下影响, 就算忍不住嘴臭也不要太大声
	
	\item 任何时候都不要着急, 着急不能解决问题, 不要当喆国王
	
	\item 输了游戏, 还有人生;赢了游戏, 还有人生.
\end{itemize}
				
				\subsection{附录: Cheat Sheet}
					见最后几页.

	\end{multicols}

	\includepdf[pages = 1-10]{../src/misc/Cheat.pdf}
\end{document}
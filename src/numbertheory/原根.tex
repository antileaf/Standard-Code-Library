\paragraph{阶} 最小的整数 $k$ 使得 $a ^ k \equiv 1 \pmod m$,记为 $\delta_m(a)$。

显然 $a$ 在阶以下次数的幂是两两不同的。

一个性质:如果 $a, b$ 均与 $m$ 互质,则
$$\delta_m(ab)=\delta_m(a)\delta_m(b) \iff \gcd\left(\delta_m(a),\delta_p(b)\right) = 1$$

另外,如果 $a$ 与 $m$ 互质,则有 $\delta_p(a^k)=\dfrac{\delta_p(a)}{\gcd\left(\delta_p(a),k\right)}$。(也就是长为 $\delta_m(a)$ 的环上一次跳 $k$ 步的周期。)

\paragraph{原根} 所有阶等于 $\varphi(m)$ 的数。

只有形如$2,\, 4,\, p^k,\, 2 p^k$($p$是奇素数)的数才有原根。并且如果 $m$ 有原根,那么原根的个数一定是 $\varphi\left(\varphi(m)\right)$ 个。

暴力找原根代码:

\begin{minted}{python}
def split(n): # 分解质因数
    i = 2
    a = []
    while i * i <= n:
        if n % i == 0:
            a.append(i)

            while n % i == 0:
                n /= i

        i += 1

    if n > 1:
        a.append(n)

    return a
    
def getg(p): # 找原根
    def judge(g):
        for i in d:
            if pow(g, (p - 1) / i, p) == 1:
                return False
        return True

    d = split(p - 1)
    g = 2

    while not judge(g):
        g += 1

    return g

print(getg(int(input())))
\end{minted}

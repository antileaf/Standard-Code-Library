\paragraph{拉格朗日反演} 如果 $f(x)$ 与 $g(x)$ 互为复合逆,则有

$$ \begin{aligned}\relax \left[x^n\right]f(x)=&\frac{1}{n}\left[x^{n-1}\right]\left(\frac{x}{g(x)}\right)^n \\
\relax \left[x^n\right]h(f(x))=&\frac{1}{n}\left[x^{n-1}\right]h'(x)\left(\frac{x}{g(x)}\right)^n\end{aligned} $$

这样可以得到复合逆的一项。如果需要 $0 \dots n$ 项的所有系数,就麻烦一些。

推导过程省略,直接上结论:

$$ \begin{aligned}
f(x) =& x \left( \sum_{k = 1} ^ n x^{n - k} \frac n k \left[ x^n \right] g^k(x) \right) ^ {- 1 / n} \\
=& \frac x {g_1} \left( \sum_{k = 1} ^ n \frac n k \left( \frac x {g_1} \right) ^ {n - k} \left[ z^n \right] \left( \frac {g(z)} {g_1} \right) ^k \right) ^ {- 1 / n}
\end{aligned} $$

$g_1$ 是 $g(x)$ 的一次项。把它提出来是为了把要开根的式子常数项化成 $1$,避免考虑 $n \bmod \varphi(p)$ 的逆元。

现在唯一的难点就在于求 $[x^n] \left( \frac {g(x)} {g_1} \right) ^ k$。

考虑更一般的问题,即 \textbf{对所有 $k \in [1, n]$,如何分别求出 $[x^n] A^k(x)$}。

引入另一个自变量 $y$,代表 $A(x)$ 的次数这一维,得到一个二元生成函数:

$$ \sum_{i} x^i \sum_{j} y^j [x^i] A^j(x) = \sum_{j} y^j A^j(x) = \frac 1 {1 - y A(x)} $$

$x^n$ 项的系数即为所求。

针对这个子问题,再考虑更一般的问题,即求 $[x^n] \frac {P(x, y)} {Q(x, y)}$。

按照 Bostan-Mori (\ref{BostanMori},第 \pageref{BostanMori} 页) 一样的思路,上下同乘 $Q(-x, y)$,又会发现分母 $Q(x, y) Q(-x, y)$ 里 $x$ 只有偶数项,只取出奇数项或者偶数项就可以把 $n$ 折半,然后递归下去即可。边界是 $\frac {P(0, y)} {Q(0, y)}$。

在这里初始时 $x$ 最高项是 $n$ 次,而 $y$ 只有一次。每步 $x$ 的最大次数都会减半,而 $y$ 的最大次数会翻倍,所以总的项数一直是 $O(n)$ 的。总的复杂度是 $O(n \log^2 n)$。

这里有一个优化:做完 $k$ 次迭代之后 $y$ 的最高次数是 $2^k$,也就是说有 $2^k + 1$ 项,在做卷积的时候就会白白多做一倍的长度。实际上可以注意到,$P(x, 0)$ 和 $Q(x, 0)$ 的 $y^0$ 项只能是 $0$ 或者 $1$,不会包含 $x$。所以可以把它提出来,这样 $y$ 这一维的长度就刚好是 $2^k$ 了。

$Q(x, 0)$ 因为每次都保留偶数项,所以常数项一直都是 $1$。$P(x, 0)$ 初始常数项也是 $1$,但在保留一次奇数项之后就一直是 $0$ 了。

设 $P(x, 0)$ 的常数项是 $k$,按照 $(Py + k) (Qy + 1) = (PQy + P + kQ)y + k$ 计算即可。

解决上面的所有子问题之后别忘了还要有一个多项式开 $n$ 次根,总代码量还是很大的。

\inputminted{cpp}{../src/math/多项式复合逆.cpp}
